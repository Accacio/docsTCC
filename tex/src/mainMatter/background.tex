
\chapter{Background}
\label{cha:background}
This chapter 
 
\section{Discrete Time Event Systems}
\label{sec:discreteEventTimeSystems}

\subsection{Automata}
\label{sec:automata}
\begin{definition}[Deterministic Automaton]
\label{def:DeterministicAutomaton}~\\  
A Deterministic Automaton, denoted by G, is a five-tuple
\[ G = (X,\Sigma,f, x_0,X_m)\]
where:

\indent X is the set of \textbf{states} \\
\indent $\Sigma$ is the finite set of \textbf{events} associated with G\\
\indent f: X $\times \Sigma \rightarrow X$ is the \textbf{transition function}  \\ 
\indent $x_0$ is the \textbf{initial state} \\
\indent $X_m \subseteq X $ is the set of \textbf{marked states}

\end{definition}
\todo{

As in \cite{cassandras2009introduction} the initial states on this thesis are
going to be identified by an arrow pointing towards them, and marked states by
double circles.

\begin{figure}[H]
  \centering
  \includegraphics[width=0.5\textwidth]{automata/example/example.tikz}
  % \includetikzfigure[width=0.5\textwidth]{automata/example/example}
  \caption{Diagram representing the Automata from example number }
\end{figure}


\subsubsection{DAOCT}

\begin{definition}[Deterministic Automaton With Outputs and Conditional
  Transitions (DAOCT)]
\label{def:daoct}~\\  
A Deterministic Automaton, denoted by DAOCT, is a nine-tuple
\[ DAOCT = (X,\Sigma,f,\lambda,R,\theta, x_0,X_f)\]
where:

\indent X is the set of \textbf{states} \\
\indent $\Sigma$ is the finite set of \textbf{events}\\
\indent $\Omega \subset \mathbb{N}_1^{m_i+m_o} $ is the set of \textbf{I/O vectors}\\
\indent f: X $\times \Sigma^\star \rightarrow X$ is the \textbf{deterministic transition function}  \\ 
\indent $\lambda : X \rightarrow \Omega$ is the \textbf{state output function} \\
\indent $R = {1,2,\dots,r}$ is the set of \textbf{path indices} \\
\indent $\theta : X \times \Sigma \rightarrow 2^R$ is the \textbf{path
  estimation function} \\
\indent $x_0$ is the \textbf{initial state} \\
\indent $X_f \subseteq X $ is the set of \textbf{final states}
\end{definition}

Identification algorithm adapted from \cite{moreira2018enhanced}
\begin{algorithm2e}
  \caption{Identification Algorithm}\label{alg:identification}
\KwIn
{%
Modified observed paths $p_i^k$, for i= 1,\dots,$r$
}
\KwOut
{%
DAOCT = $($\XSet,\SigmaSet,\OmegaSet,\ffunction,\lambdafunction,\RSet,\thetafunction,\xZero,\XfSet$)$
}
\BlankLine
Create an initial state $x_0$, and define $\lambda(x_0) = \tilde{\lambda}(x_0) =
y_{1,1}$

$X = \{ x_0\}, X_f = \emptyset, R = \emptyset$

\For{$i = 1$ \KwTo $r$}
{
  $R = R \cup \{ i \}$
  
\For{$j = 1$ \KwTo $l_i - 1$}
{
  Find the State $x \in X $ such that $\tilde{\lambda}(x) = y_{i,j+1}$

  \eIf{$\tilde{\lambda}(s) \neq y_{i,j+1}$ for all $ s \in X$}
  { Create state $x^\prime$ and define $\tilde{\lambda}(x^\prime) = y_{i,j+1}$

$X = X \cup \{ x^\prime\}$

$\lambda(x^\prime) = \tilde{\lambda_l}(x^\prime)$

}
{
  Find $x^\prime \in X$ such that $\tilde{\lambda}(x^\prime) = y_{i,j+1}$
}
$f(x,\sigma_{i,j}) = x^\prime$

Add $i$ to $\theta(x,\sigma_{i,j})$

\If{$j = l_i - 1$}
{
  $X_f = X_f \cup \{x^\prime\}$
}
}
}
\end{algorithm2e}



\begin{figure}[H]
  \centering
  \includegraphics[width=\textwidth]{automata/daoct/example.tikz}
  % \includetikzfigure[width=0.5\textwidth]{automata/example/example}
  \caption{Diagram representing the DAOCT from example number }
\end{figure}

\subsection{Petri Nets}
\label{sec:petriNets}
\subsubsection{Control Interpreted Petri Net}
% pag 65 discrete continuos and hybrid \cite{david2005discrete} 

\section{Model of automata}

\section{Identification}
  formas de identificação
  algoritmos

}
\section{Petri Nets}
\label{sec:petriNets}


Adapted from \cite{david1989grafcet}
\begin{figure}[H]
  \centering
  \includegraphics[width=0.8\textwidth]{cipnExample/scheme.tikz}
  \caption[cipnexample]{Example of System to be controlled by the Petri Net}
  \label{fig:cipnexamplescheme}
\end{figure}

\pagebreak
\begin{figure}[H]
  \centering
  \includegraphics[width=0.8\textwidth]{cipnExample/cipn.tikz}
  \caption[cipnexample]{Example of Control Interpreted Petri Net to control
    system in \autoref{fig:cipnexamplescheme}}
  \label{fig:cipnexample}
\end{figure}

\begin{table}[htbp]
\caption{Control Interpreted Petri Net Example Places.}
\centering
\begin{tabular}{M{5cm}M{10cm}}
Places & Meaning\\
\hline
\hyperlink{cipnExampleNet:p0m1}{\hypertarget{cipnExampleTable:p0m1}{$p_{0}$}} & System Stopped\\
\hyperlink{cipnExampleNet:p1}{\hypertarget{cipnExampleTable:p1}{$p_{1}$}} & R (Car Moving to the Right)\\
\hyperlink{cipnExampleNet:p2}{\hypertarget{cipnExampleTable:p2}{$p_{2}$}} & Open (Container Opened)\\
\hyperlink{cipnExampleNet:p3}{\hypertarget{cipnExampleTable:p3}{$p_{3}$}} & L (Car Moving to the Left)\\
\end{tabular}
\end{table}

\begin{table}[H]
\caption{Control Interpreted Petri Net Example Transitions.}
\centering
\begin{tabular}{M{5cm}M{10cm}}
Transitions & Meaning\\
\hline
\hyperlink{cipnExampleNet:t0}{\hypertarget{cipnExampleTable:t0}{$t_{0}$}} & \(\uparrow\) m (filling request)\\
\hyperlink{cipnExampleNet:t1}{\hypertarget{cipnExampleTable:t1}{$t_{1}$}} & \(\uparrow\) b (Right Limit Switch)\\
\hyperlink{cipnExampleNet:t2}{\hypertarget{cipnExampleTable:t2}{$t_{2}$}} & \(\uparrow\) p (Car is Full)\\
\hyperlink{cipnExampleNet:t3}{\hypertarget{cipnExampleTable:t3}{$t_{3}$}} & \(\uparrow\) a (Left Limit Switch)\\
\end{tabular}
\end{table}

\usetikzlibrary{arrows,shapes,circuits.plc.ladder,external}

\begin{figure}[H]
  \centering
  \includegraphics{cipnExample/cipnLadder.tikz}
  \caption[cipnexample]{Example of Control Interpreted Petri Net converted to Ladder.}
  \label{fig:cipnexampleLadder}
\end{figure}

\begin{figure}[H]
    \centering
    \begin{subfigure}[t]{0.5\textwidth}
      \centering
        \includetikzfigure[width=\textwidth]{communicationPlcPN/communicationPlcPN}
        \caption{Petri Net on PLC 1.}
        \label{fig:communicationPlcPN}
    \end{subfigure}%
    ~ 
    \begin{subfigure}[t]{0.5\textwidth}
        \centering
        \includetikzfigure[width=\textwidth]{communicationPlcPN/communicationPlcPN1}
  \caption{Petri Net on PLC 2.}
  \label{fig:communicationPlcPN1}
    \end{subfigure}
    \caption{Example of Petri Net divided between 2 PLCs.}
\end{figure}


\begin{figure}[H]
    \centering
    \begin{subfigure}[t]{0.45\textwidth}
        \centering
        \includegraphics{communicationPlcPN/communicationPlcPNLadder.tikz}
        \caption{Ladder Logic on PLC 1.}
        \label{fig:communicationPlcPN}
    \end{subfigure}%
\hfill
    \begin{subfigure}[t]{0.45\textwidth}
        \centering
        \includegraphics{communicationPlcPN/communicationPlcPN1Ladder.tikz}
  \caption{Ladder Logic on PLC 2.}
  \label{fig:communicationPlcPN1}
    \end{subfigure}
    \caption{Example of Petri Net divided between 2 PLCs.}
\end{figure}
  

\autoref{fig:communicationPlcPN}

\usetikzlibrary{patterns}
\begin{figure}[H]
  \centering
  \includegraphics[width=0.5\textwidth]{vennDiagramLanguages.tikz}
  \caption{Venn diagram showing relations between $L_{Orig}$, $L_{OrigNI}$,
    $L_{Obs}$, $L_{Exc}$ and $L_{Iden}$}
\end{figure}


%%% Local Variables:
%%% mode: latex
%%% TeX-master: "../monografia.tex"
%%% End: