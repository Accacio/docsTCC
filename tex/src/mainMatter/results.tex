\chapter{Results}
\label{cha:results}

\section{DAOCT}
\label{sec:results_daoct}

First to show the algorithm developed works, the example data from
\cite{moreira2018enhanced} will be used.
\todo{comentar os grafos}
\begin{figure}[H]
  \centering
  \includetikzfigure[width=\textwidth]{example/examplek1}
  \caption{DAOCT of example 2 from }
  \label{fig:exceedingLangExample}
\end{figure}

\begin{figure}[H]
  \centering
  \includetikzfigure[width=\textwidth]{example/examplek2}
  \caption{DAOCT of example 2 from }
  \label{fig:exceedingLangExample}
\end{figure}

\begin{figure}[H]
  \centering
  \begin{tikzpicture}
        % \draw[help lines,xstep=1,ystep=1] (0,0) grid (10,10);
        % \foreach \x in {0,1,...,10} { \node [anchor=north] at (\x,0) {\x}; }
        % \foreach \y in {0,1,...,10} { \node [anchor=east] at (0,\y) {\y}; }

        % Belt 
        \draw[very thick] (1,1) -- (9,1);
        \draw[very thick] (1,2) -- (9,2);
        \draw[very thick] (1,1.5) circle(0.5);
        \draw[thick,fill] (1,1.5) circle(0.05);
        \draw[very thick] (9,1.5) circle(0.5);
        \draw[thick,fill] (9,1.5) circle(0.05);

        % Sensors
        \newcommand{\beltSensor}[3]{
          \draw[very thick] (#1-0.25,#2) rectangle ++ (0.5,1);
          \draw[dashed,very thick] (#1,#2) -- (#1,2);
          \draw (#1,#2+1.5) node {#3};
        }
       \beltSensor{2.5}{4}{$S_1$}
       \beltSensor{5}{4}{$S_2$}
       \beltSensor{7.5}{4}{$S_3$}

        % Box
        \newcommand{\beltBox}[3]{
          \draw[very thick] (#1,#2) rectangle ++ (#3,#3);
          \draw[very thick] (#1,#2) -- ++ (#3,#3);
          \draw[very thick] (#1,#2+#3) -- ++ (#3,-#3);
        }
        \beltBox{3.5}{2}{1} 
        % \draw [fill,white,fill opacity=0.7,draw=none] (0.1,0.82) rectangle  (0.2,0.96);
        % \draw [red,thick] ([shift=(0:0.03)]0.15,0.9) arc (0:180:0.03);
        % \draw[black,->,>=stealth,very thick] (0.15,0.85) -- ++(0,0.1);
        \draw [->,>=stealth,thick] ([shift=(225:0.25)]9,1.5) arc (225:45:0.25);
        \draw [->,>=stealth,thick] ([shift=(225:0.25)]1,1.5) arc (225:45:0.25);

        % \beltBox{1.5}{2}{0.5} 
        % \beltBox{4.5}{2}{0.5} 
        % \beltBox{7.5}{2}{0.5} 
    \end{tikzpicture}
  \caption{Input\slash Output Process model}
    \label{fig:ioProcModel}
\end{figure}




% \begin{figure}[H]
%   \centering
%   \includegraphics[width=0.5\textwidth]{exceedingLanguage/example/exceedingLanguage-daoct-ndaao_k2_n7.pdf}
%   \caption{Cardinality of the exceeding language of the DAOCT (o) and NDAAO
%     ($\times$) models. $k = 1$, and $1 \leq n \leq 7$}
%   \label{fig:exceedingLangExample}
% \end{figure}


% Comparing the results of the \autoref{fig:exceedingLangExample}  with the
% example 3 from \cite{moreira2018enhanced}, we can
% observe that the exceeding language for the DAOCT model drops. This is caused by
% how the acquisition works, in this work, the plant is considered a black box, so
% instead of feeding the algorithm with the
% paths, the raw data is given, and the paths are calculated using the first
% IO\_Vector as the initial state and once this initial state is repeated other
% path 
% is created, resulting on 4 paths instead of 3. This change, can result in a
% smaller path, with no loops, diminishing the exceeding language.


% \section{Manufacture System}
% \label{sec:results_system}

% \begin{figure}[H]
%   \centering
%   \includegraphics[width=0.5\textwidth]{results/all/exceedingLanguage-daoct-ndaao_k1_n7.pdf}
%   \caption{graph}
% \end{figure}

% \begin{figure}[H]
%   \centering
%   \includegraphics[width=0.5\textwidth]{results/all/exceedingLanguage-daoct-ndaao_k2-3-7_n25.pdf}
%   \caption{graph}
% \end{figure}

% \todo{Choosing the IO\_Vector with the greatest repetition ratio as $x_0$}

% \begin{figure}[H]
%   \centering
%   \includegraphics[width=0.5\textwidth]{results/all/best/exceedingLanguage-daoct-ndaao_k1_n7.pdf}
%   \caption{graph}
% \end{figure}


% \begin{figure}[H]
%   \centering
%   \includegraphics[width=0.5\textwidth]{results/all/best/exceedingLanguage-daoct-ndaao_k2-3-7_n25.pdf}
%   \caption{graph}
% \end{figure}

% Removing I\_MAG1EMPT and I\_MAG2EMPT

% \begin{figure}[H]
%   \centering
%   \includegraphics[width=0.5\textwidth]{results/all-2_5/exceedingLanguage-daoct-ndaao_k1_n7.pdf}
%   \caption{graph}
% \end{figure}

% \todo{Choosing the IO\_Vector with the greatest repetition ratio as $x_0$}

% \begin{figure}[H]
%   \centering
%   \includegraphics[width=0.5\textwidth]{results/all-2_5/best/exceedingLanguage-daoct-ndaao_k1_n7.pdf}
%   \caption{graph}
% \end{figure}

% As we can see, the exceedingLanguage raises




%%% Local Variables:
%%% mode: latex
%%% TeX-master: "../monografia"
%%% End:
