\chapter{Conclusion}
\label{cha:conclusion}
As proposed in the introduction, this work presents a methodology and tools for
control, observation and identification of \DESs{} in order to have a model to be
used for fault detection. In this chapter a brief retrospective of the process
of preparation of the methodology and the tools presented is made. And then the
conclusions are drawn based on the results generated by the application of
the methodology
to identify a didactic manufacturing system.
% The control implementation was based on the
% method shown in \cite{moreira2013bridging} and the identification model based on \cite{moreira2018enhanced}.
% This identification method based on the
% observation of the fault-free behaviour of the system
% acquisition , from its conception.

Through the process of preparation of the methodology and tools, some issues
were found, and they are compiled in this chapter. Some
workarounds are proposed as new approaches to solve these issues in future works.

\section{Concluding Remarks}
\label{sec:concludingRemarks}
So, in this work we could see the methodology to control, observe and identify a
\DES{}. First, the control logic was created using a \CIPN, and then implemented
in \LD{} to be used in a Siemens \PLC{} (\Autoref{cha:control}). After this, the observation of its
inputs\slash outputs was made using data log function blocks that saved the data in
\verb|.csv| files, and finally these \verb|.csv| files were input in the
identification algorithm generating a \DAOCT{} model (\Autoref{cha:ident}). In
\Autoref{cha:results} we could see that if the system was observed for a long
time and the initial state of observation was well-chosen, then the \DAOCT{} is
a good candidate for modelling, if the aim of this modelling is fault-detection.
The fact that the exceeding language of the \DAOCT{} model drops to $0$ more
rapidly than other models, with a smaller value of the variable $k$, proves that it is less resource intensive than the
others, even for relatively big
systems, with more than $60$ inputs\slash outputs with concurrent behaviour.


\section{Further Work}
An issue found in the implementation of the control is the use of \LD{} to
program the logic. Although \LD{} is very used
in the industry, as it is a visual language, it creates a difficulty for the
automation of the conversion from Petri net. An approach that can be used in
future works would be to represent the Petri net in a text format, Petri net
markup language for instance (presented in \cite{weber2003petri}), and create a
tool that automatically converts this file to a text based language standardised
by the IEC 61131-1, \IL{} or \ST{}. Since \IL{} is less and less used, \ST{}
would be the logical choice. Using a text based language increases portability
of the code and it helps the development, since version control can be used in
text files, allowing the collaboration of multiple people to edit the code if
needed, and track who made the changes and when, increasing the maintainability
of the code.



Another issue was about the observation. Although the acquisition of
inputs\slash outputs using data logs and saving the data in batches on
\verb|.csv| files can be used for the identification process, for
fault-detection it is not optimal to use this approach, a better one would be to
acquire the data in real time, by using some API, snap7 for example, or using
\SCADA{} protocols.
% But if we use in a future work the function block created to log
% the data in \Autoref{cha:ident}, the \emph{LOGDATA} block, it is recommended to
% optimise its contents. Some refactoring on the logic could be made, increasing its
% speed and removing some unnecessary variables that may be present.

As shown in \Autoref{cha:results}, the didactic manufacturing system used for the
experiments have a considerable concurrent behaviour, affecting the identified
model, on the number of states and extracted paths. An approach that can be made
in other works is to divide the observation of the system in its modules, and
compare the multiple models generated by the identification algorithm with the
one using the observation of the complete system.

Another issue shown in \Autoref{cha:results}, is the choice of the first
vector to be used as initial state in the identification algorithm. Here we
propose for future works a study on how to find the optimal vector. Two
scenarios could be considered: the first one taking a grey box approach, where some behaviour
is previously known, by a simple description of the function of the system and another considering a black box approach.

Another proposition for a future work is made in \Autoref{cha:results}. Instead
of using an identification model based on the observation of inputs\slash
outputs of the system, an alternative would be to create and use a model that uses the observation of
the events of the system.


% to extract the paths and events of the system and to identify the system
% behaviour,
% a way out would be to observe the events and use them to identify the
% system. Of course the \DAOCT{} model would not be fit for it, and another model
% should be developed. This way, probably the problem with the choice of the
% initial state of the system would cease to exist.


%%% Local Variables:
%%% mode: latex
%%% TeX-master: "../monografia"
%%% End:
