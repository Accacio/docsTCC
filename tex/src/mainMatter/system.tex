
\chapter{System}
\label{chap:system}
In this chapter, the system to be control and identified is presented.
The system is a didactic manufacture system assembled from submodules fabricated
by Christiani\footnote{All images from the Christiani modules are present on its sales
  catalog, available at \url{www.christiani.de}. All rights are reserved to Christiani.}, a German
constructor specialised in Mechatronic Systems and Industry Models to
didactic ends.
This manufacture system is located in the \LCA, situated in the \UFRJ. This
system is normally used for the under-graduated studies about Industrial
Automation and control of \DESs, and also for data acquisition in some
bachelor\slash master\slash doctorate thesis, as this one.

This manufacture system is a cube assembly system, where the different cube
halves shown in \autoref{fig:cubeHalves} are put together to form cubes.
\begin{figure}[H]
  \centering
  \includegraphics[width=0.55\textwidth]{maquete/pieces/workPieces.jpg}
  \caption{Cube halves.}
  \label{fig:cubeHalves}
\end{figure}

The pieces can be of two materials, metal or plastic, and the plastic ones can
be white or black.
The
permutation of cube halves needed to form a cube is selected via a type of
sorting, selecting the type of piece by material and colour. The assembled cubes
are then stored. In order to perform these tasks (sorting, handling, assembling
and stocking), 6 Units from Christiani manufacturer are used. These units can be
seen in \autoref{fig:units}.

\begin{figure}[H]
\begin{subfigure}[t]{0.35\textwidth}
  \centering
  \includegraphics[width=\textwidth]{maquete/mag/mag.jpg}
  \caption{Magazine Unit}
\end{subfigure}
\hfill
\begin{subfigure}[t]{0.35\textwidth}
  \centering
  \includegraphics[width=\textwidth]{maquete/esteira/esteira.jpg}
  \caption{Conveyor Belt.}
\end{subfigure}

\begin{subfigure}[t]{0.35\textwidth}
  \centering
  \includegraphics[width=\textwidth]{maquete/sensores/sensores.jpg}
  \caption{Sorting Unit.}
\end{subfigure}
\hfill
\begin{subfigure}[t]{0.35\textwidth}
  \centering
  \includegraphics[width=\textwidth]{maquete/braco/braco.jpg}
  \caption{Handling Unit.}
\end{subfigure}

\begin{subfigure}[t]{0.35\textwidth}
  \centering
  \includegraphics[width=\textwidth]{maquete/prensa/prensa.jpg}
  \caption{Assembly Unit.}
\end{subfigure}
\hfill
\begin{subfigure}[t]{0.35\textwidth}
  \centering
  \includegraphics[width=\textwidth]{maquete/elevador/elevador.jpg}
  \caption{Storage Unit.}
\end{subfigure}
  \caption{Units of the Manufacture System.}
  \label{fig:units}
\end{figure}

In the next sections each unit and their Inputs\slash Outputs will be detailed.

\begin{observation}
What is described in the next sections as an input of a certain module, it is
considered as an output for the controller and vice versa.  
\end{observation}

\section{Magazine Unit}
\label{sec:magazine}
The magazine is a unit with the objective to stock the cube halves to be used.
There are 2 types of magazines, one to stock pieces without connection pins (the pins shown in \autoref{fig:cubeHalves}) and another to
 stock pieces with those pins inserted, they can stack 10 and 8 pieces
 respectively. They will be denominated \verb| MAG 1| and \verb|MAG 2|.
Each magazine has a cylinder and a presence button. The cylinder serves to extract a
piece from the bottom of the stack, and the button to know if the stack is empty
or not. These cylinders have 2 inputs and 2
outputs. The inputs they are used to extend and retract the cylinders (if they
are set to $true$) and the
outputs are used to know if the cylinders are extended or retracted, the output
is equal to $true$ if the respective condition is fulfilled. These
inputs  are called in this work \verb|Extend MAG 1/2 Cylinder| and
\verb|Retract MAG 1/2 Cylinder|, and the outputs are called \verb|MAG 1/2 Cylinder Extended| and
\verb|MAG 1/2 Cylinder Retracted|. The presence button of each magazine outputs a $true$
value if the stack is empty and $false$, otherwise. Thus this presence buttons
are called in this work \verb|MAG 1/2 Empty|, their localisation on the magazine
can be seen in \autoref{fig:magazine2}. 
\begin{figure}[H]
  \centering
  \begin{tikzpicture}
    \node[anchor=south west,inner sep=0] (image) at (0,0) {
      \includegraphics[width=8cm]{maquete/mag/30540_4.jpg}
    };
    % \draw[red,ultra thick,rounded corners] (0,0) rectangle (9.4,6.2);
    \begin{scope}[x={(image.south east)},y={(image.north west)}]
        % \draw[help lines,xstep=.1,ystep=.1] (0,0) grid (1,1);
        % \foreach \x in {0,1,...,9} { \node [anchor=north] at (\x/10,0) {0.\x}; }
        % \foreach \y in {0,1,...,9} { \node [anchor=east] at (0,\y/10) {0.\y}; }
      \draw[red] (0.85,0.55) node {\textbf{Cylinder}};
      \draw[<-,>=stealth,red,very thick] (0.8,0.5) -- (0.67,0.37);
      \draw[red] (0.75,0.7) node {\textbf{Presence Button}};
      \draw[red,very thick, rounded corners] (0.25,0.35) rectangle (0.35,0.45);
      \draw[<-,>=stealth,red,very thick] (0.6,0.65) -- (0.37,0.45);
      \end{scope}
  \end{tikzpicture}
  \caption{Magazine Unit}
  \label{fig:magazine2}
\end{figure}

% only presence button
% \begin{figure}[H]
%   \centering
%   \begin{tikzpicture}
%     \node[anchor=south west,inner sep=0] (image) at (0,0) {
%       \includegraphics[width=8cm]{maquete/mag/30540_4.jpg}
%     };
%     % \draw[red,ultra thick,rounded corners] (0,0) rectangle (9.4,6.2);
%     \begin{scope}[x={(image.south east)},y={(image.north west)}]
%         % \draw[help lines,xstep=.1,ystep=.1] (0,0) grid (1,1);
%         % \foreach \x in {0,1,...,9} { \node [anchor=north] at (\x/10,0) {0.\x}; }
%         % \foreach \y in {0,1,...,9} { \node [anchor=east] at (0,\y/10) {0.\y}; }
%       \draw[red] (0.75,0.5) node {\textbf{Presence Button}};
%       \draw[red,very thick, rounded corners] (0.25,0.35) rectangle (0.35,0.45);
%       \draw[->,red,very thick] (0.5,0.5) -- (0.37,0.45);
%       \end{scope}
%   \end{tikzpicture}
%   \caption{Magazine Unit}
% \end{figure}

\section{Conveyor Belt}
\label{sec:magazine}
The conveyor belt is a unit with the objective of transporting the pieces from a
unit to another. It has 2 inputs and 1 output. The inputs are used to turn the
belt on, but each input makes it turn in a direction or the other. The output is
the generated by a presence sensor located in on extremity of the belt (see
\Autoref{fig:conveyorBelt}), it is equal to $true$ if there is a piece in front
of it and $false$ otherwise. The directions of the movement of the pieces is
denominated Forward if it is going towards the presence sensor and reverse if
not. Thus the names given to the inputs that generate this movements are
\verb|Conveyor Belt Forward| and \verb|Conveyor Belt Reverse|.
And the input is called \verb|Proximity Sensor End of Conveyor Belt|.

\begin{figure}[H]
  \centering
  \begin{tikzpicture}
    \node[anchor=south west,inner sep=0] (image) at (0,0) {
      \includegraphics[trim={0 6cm 0 5cm},clip,width=8cm]{maquete/esteira/40778_3.jpg}
    };
    % \draw[red,ultra thick,rounded corners] (0,0) rectangle (9.4,6.2);
    \begin{scope}[x={(image.south east)},y={(image.north west)}]
        % \draw[help lines,xstep=.1,ystep=.1] (0,0) grid (1,1);
        % \foreach \x in {0,1,...,9} { \node [anchor=north] at (\x/10,0) {0.\x}; }
        % \foreach \y in {0,1,...,9} { \node [anchor=east] at (0,\y/10) {0.\y}; }
        \draw [->,>=stealth,red, very thick](0.9,0.75) -- ++ (-0.8,0.0);
        \draw [red] (0.5,0.85) node {Forward};
        \draw [->,>=stealth,red, very thick](0.1,0.95) -- ++ (0.8,0.0);
        \draw [red](0.5,1.05) node {Reverse};
        
        \draw [red](0.5,0.0) node {Presence Sensor};
        \draw [red,very thick,rounded corners](0.05,0.45) rectangle (0.12,0.65);
        \draw [->,>=stealth,red, very thick](0.1,0.4) -- (0.3,0.1);
      \end{scope}
  \end{tikzpicture}
  \caption{Conveyor Belt}
  \label{fig:conveyorBelt}
\end{figure}

% no sensor
% \begin{figure}[H]
%   \centering
%   \begin{tikzpicture}
%     \node[anchor=south west,inner sep=0] (image) at (0,0) {
%       \includegraphics[trim={0 6cm 0 5cm},clip,width=8cm]{maquete/esteira/40778_3.jpg}
%     };
%     % \draw[red,ultra thick,rounded corners] (0,0) rectangle (9.4,6.2);
%     \begin{scope}[x={(image.south east)},y={(image.north west)}]
%         % \draw[help lines,xstep=.1,ystep=.1] (0,0) grid (1,1);
%         % \foreach \x in {0,1,...,9} { \node [anchor=north] at (\x/10,0) {0.\x}; }
%         % \foreach \y in {0,1,...,9} { \node [anchor=east] at (0,\y/10) {0.\y}; }
%         \draw [->,>=stealth,red, very thick](0.9,0.7) -- (0.1,0.7);
%         \draw [red] (0.5,0.8) node {Forward};
%         \draw [->,>=stealth,red, very thick](0.1,0.1) -- (0.9,0.1);
%         \draw [red](0.5,0.0) node {Reverse};
%       \end{scope}
%   \end{tikzpicture}
%   \caption{Conveyor Belt}
% \end{figure}

\section{Sorting Unit}
\label{sec:sortingUnit}
As the name says, the sorting unit serves to sort the pieces. 
\begin{figure}[H]
  \centering
  \begin{tikzpicture}
    \node[anchor=south west,inner sep=0] (image) at (0,0) {
      \includegraphics[trim={0 0 0 0},clip,width=8cm]{maquete/sensores/69511_2.jpg}
    };
    % \draw[red,ultra thick,rounded corners] (0,0) rectangle (9.4,6.2);

    \begin{scope}[x={(image.south east)},y={(image.north west)}]
        \draw[help lines,xstep=.1,ystep=.1] (0,0) grid (1,1);
        \foreach \x in {0,1,...,9} { \node [anchor=north] at (\x/10,0) {0.\x}; }
        \foreach \y in {0,1,...,9} { \node [anchor=east] at (0,\y/10) {0.\y};  }
        \draw[red,ultra thick,rounded corners] (0.15,0.4) rectangle ++ (0.15,0.1);
        \draw[red] (0.1,0.1) node {\textbf{Left}};
        \draw[->,>=stealth,red, very thick] (0.2,0.38) -- (0.1,0.15);
        \draw[magenta,ultra thick,rounded corners] (0.35,0.4) rectangle ++ (0.15,0.1);
        \draw[magenta] (0.1,0.8) node {\textbf{Center}};
        \draw[->,>=stealth,magenta, very thick] (0.4,0.52) -- (0.2,0.75);
        \draw[cyan,ultra thick,rounded corners] (0.53,0.4) rectangle ++ (0.15,0.1);
        \draw[cyan] (0.7,0.1) node {\textbf{Right}};
        \draw[->,>=stealth,cyan, very thick] (0.65,0.38) -- (0.7,0.15);
      \end{scope}
  \end{tikzpicture}
  \caption{Sorting Unit - Identification}
  \label{fig:sortIden}
\end{figure}
\begin{figure}[H]
  \centering
  \begin{tikzpicture}
    \node[anchor=south west,inner sep=0] (image) at (0,0) {
      \includegraphics[trim={0 0 0 0},clip,width=8cm]{maquete/sensores/69511_3.jpg}
    };
    % \draw[red,ultra thick,rounded corners] (0,0) rectangle (9.4,6.2);

    \begin{scope}[x={(image.south east)},y={(image.north west)}]
        % \draw[help lines,xstep=.1,ystep=.1] (0,0) grid (1,1);
        % \foreach \x in {0,1,...,9} { \node [anchor=north] at (\x/10,0) {0.\x}; }
        % \foreach \y in {0,1,...,9} { \node [anchor=east] at (0,\y/10) {0.\y};  }
        \draw[red,ultra thick,rounded corners] (0.15,0.4) rectangle ++ (0.15,0.1);
        \draw[red] (0.1,0.1) node {\textbf{Left}};
        \draw[->,>=stealth,red, very thick] (0.2,0.38) -- (0.1,0.15);
        \draw[magenta,ultra thick,rounded corners] (0.35,0.4) rectangle ++ (0.15,0.1);
        \draw[magenta] (0.1,0.8) node {\textbf{Center}};
        \draw[->,>=stealth,magenta, very thick] (0.4,0.52) -- (0.2,0.75);
        \draw[cyan,ultra thick,rounded corners] (0.53,0.4) rectangle ++ (0.15,0.1);
        \draw[cyan] (0.7,0.1) node {\textbf{Right}};
        \draw[->,>=stealth,cyan, very thick] (0.65,0.38) -- (0.7,0.15);
      \end{scope}
  \end{tikzpicture}
  \caption{Sorting Unit - Discharge}
  \label{fig:sortDisc}
\end{figure}

\section{Handling Unit}
\label{sec:handlingUnit}

\section{Assembly Unit}
\label{sec:assemblyUnit}

\section{Storage Unit}
\label{sec:storageUnit}
\begin{figure}[H]
  \centering
  \begin{tikzpicture}
    \node[anchor=south west,inner sep=0] (image) at (0,0) {
      \includegraphics[width=8cm]{maquete/elevador/69523_3.jpg}
    };
    % \draw[red,ultra thick,rounded corners] (0,0) rectangle (9.4,6.2);
    \begin{scope}[x={(image.south east)},y={(image.north west)}]
        % \draw[help lines,xstep=.1,ystep=.1] (0,0) grid (1,1);
        % \foreach \x in {0,1,...,9} { \node [anchor=north] at (\x/10,0) {0.\x}; }
        % \foreach \y in {0,1,...,9} { \node [anchor=east] at (0,\y/10) {0.\y}; }
      \draw[red] (1,0.5) node {\textbf{Right}};
      \draw[red] (0,0.5) node {\textbf{Left}};
      \draw[red] (0.5,1) node {\textbf{Top}};
      \draw[red] (0.5,0) node {\textbf{Bottom}};
      \end{scope}
  \end{tikzpicture}
  \caption{Storage Unit}
\end{figure}


%%% Local Variables:
%%% mode: latex
%%% TeX-master: "../monografia"
%%% End:
