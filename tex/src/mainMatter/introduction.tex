
\chapter{Introduction}
In a world where the majority of the population lives in industrial societies,
and machines take part on the bulk of the production of almost all goods: from
food to cosmetics and drugs, from toothbrushes to automobiles, 
a well-paced throughput it is crucial, and any non expected halt on the
production or change can be disastrous, producing sometimes multimillionaire debts,
provoking a snowball effect, affecting the economy and consequentially the welfare of the society.

A diverse number of causes of the halt or change of the throughput can be
accounted for. Some are as simple as a power outage, or a component malfunction,
but nowadays there are other players. As the industry \textsl{walks}, or even
better \textsl{runs}, towards the so called Fourth Industrial Revolution, the industry
urges the use of \textit{connected sensors}, since the Internet of Things is the
fashion these days, but the concerns about cyber security are now and again neglected.   
So hackers can infiltrate the system, and depending of the infrastructure
halt or change somehow the production throughput.

Cyber security is not the theme of this thesis but its theme is another important concern to a
well-paced throughput, failure detection.

As great part of the manufacture facilities uses conveyor
belts, pneumatic cylinders and digital sensors, it is very common to see \PLCs



\PLC 




\acr{oi}{oi}{oi}
\oi





khe  the most simple as  In order to prevent these effects lots of 
\todo{ objetivo mostrar que o metodo pode funcionar 
com comportamento paraelo mostrar a escabilidade
}
\doing{the objective of this thesis is to show that the \DAOCT model works with
  systems that presents parallel behavior dividing themand can be used to scalability }
\section{Thesis Outline}
\label{sec:thesisOutline}






%%% Local Variables:
%%% mode: latex
%%% TeX-master: "../monografia.tex"
%%% End:
