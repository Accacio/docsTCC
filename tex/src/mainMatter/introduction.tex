
\chapter{Introduction}
In a world where the majority of the population lives in industrial societies,
and machines take part on the bulk of the production of almost all goods: from
food to cosmetics and drugs, from toothbrushes to automobiles, 
a well-paced throughput it is crucial, and any non expected halt on the
production or change can be disastrous, producing sometimes multimillionaire debts,
provoking a snowball effect, affecting the economy and consequentially the welfare of the society.

A diverse number of causes of the halt or change of the throughput can be
accounted for. Some are as simple as a power outage, or a component malfunction,
but nowadays there are other players. As the industry \textsl{walks}, or even
better \textsl{runs}, towards the so called Fourth Industrial Revolution, the industry
urges the use of \textit{connected sensors}, since the Internet of Things is the
fashion these days, but the concerns about cyber security are now and again neglected.   
So hackers can infiltrate the system, and depending of the infrastructure
halt or change somehow the production throughput.

Cyber security is not the theme of this thesis but its theme is another important concern to a
well-paced throughput, fault detection.

As great part of the manufacture facilities uses discrete sensors and actuators,
as conveyor belts, pneumatic cylinders, limit switches and proximity sensors, it is very common to see 
\PLCs controlling those plants. And when a system is ruled by discrete events
and also its states are discrete it can be modeled by Discrete Event
Systems.  

On the literature, we can find an expressive number of articles using Discrete
Event Systems for identification, fault detection and fault
diagnosis. \cite{veras2018distributed,cabral2017synchronous,kumar2014comments,klein2005fault}  
can be used as examples.

This work is based on one of this articles, \cite{moreira2018enhanced}, that
develops an algorithm to identify a model of the system using just its inputs
and outputs, using a black box approach, also seen in other works as
\cite{klein2005fault} and \cite{roth2009fdi}. This identified model can be later
used to detect faults on the system.

The objective is to apply the identification algorithm shown in
\cite{moreira2018enhanced} in a Didactic Manufacture
System with a strong parallel behavior and a moderate number of inputs and
outputs ( over 40) and show that this algorithm can be used
dividing the parallel subsystems, so we can achieve scalability. 


During this work all steps from the conception of the control of the system to
its identification. So, in order to ease the path throughout this work we have
in the next section its outline.

\section{Thesis Outline}
\label{sec:thesisOutline}






%%% Local Variables:
%%% mode: latex
%%% TeX-master: "../monografia.tex"
%%% End:
