

\newglossarystyle{dottedlocations}{%
   \glossarystyle{list}%
   \renewcommand*{\glossaryentryfield}[5]{%
   \item[\glsentryitem{##1}\glstarget{##1}{##2}] ##3, p %
       \unskip\leaders\hbox to 2.9mm{} ##5}%
   \renewcommand*{\glsgroupskip}{}%
}

 \newglossarystyle{acronyms}{%
 % put the glossary in the itemize environment:
 \renewenvironment{theglossary}%
   {\begin{tabbing}}{\end{tabbing}}%
 % have nothing after \begin{theglossary}:
 \renewcommand*{\glossaryheader}{}%
 % have nothing between glossary groups:
 \renewcommand*{\glsgroupheading}[1]{}%
 \renewcommand*{\glsgroupskip}{}%
 % set how each entry should appear:
 \renewcommand*{\glossentry}[2]{%
 \glstarget{##1}{\glossentryname{##1}}\\% \kill % the entry name
 \=\glossentrysymbol{##1}% the symbol in brackets
 \space \glossentrydesc{##1},% the description
 \space p. ##2\\% the number list in square brackets
 }%
 % set how sub-entries appear:
 \renewcommand*{\subglossentry}[3]{%
   \glossentry{##2}{##3}}%
}


 \newglossarystyle{symbols}{%
 % put the glossary in the itemize environment:
 \renewenvironment{theglossary}%
   {\begin{tabbing}}{\end{tabbing}}%
 % have nothing after \begin{theglossary}:
 \renewcommand*{\glossaryheader}{}%
 % have nothing between glossary groups:
 \renewcommand*{\glsgroupheading}[1]{}%
 \renewcommand*{\glsgroupskip}{}%
 % set how each entry should appear:
 \renewcommand*{\glossentry}[2]{%
 \glstarget{##1}{\glossentryname{##1}}\\% \kill % the entry name
 \=\textbf{\glossentrysymbol{##1}}% the symbol in brackets
 
 \space \glossentrydesc{##1},% the description
 \space p. ##2\\% the number list in square brackets
 }%
 % set how sub-entries appear:
 \renewcommand*{\subglossentry}[3]{%
   \glossentry{##2}{##3}}%
 }


\acr{DAOCT}{DAOCT}{Deterministic Automaton
  with Outputs and Conditional Transitions}
\acr{ECA}{ECA}{Engenharia de controle e Automação}
\symbl{OmegaSet}{$\Omega$}{Set of IO vectors}


%%% Local Variables:
%%% mode: latex
%%% TeX-master: "./monografia.tex"
%%% End: