\newcommandx\acr[5][4=,5=]{
  \ifthenelse{\equal{#5}{}}
  {
    \acrSing{#1}{#2}{#3}
  }
  {
    \acrPl{#1}{#2}{#3}{#4}{#5}
  }
  } 

\newcommand{\acrSing}[3]{\newacronym{#1}{#2}{#3}
  \expandafter\newcommand\csname #1\endcsname{\gls{#1}}}

\newcommand{\acrPl}[5]{
  \newacronym[plural=#4,firstplural=#5 (#4)]{#1}{#2}{#3}
  \expandafter\newcommand\csname #1\endcsname{\gls{#1}}
  \expandafter\newcommand\csname #4\endcsname{\glspl{#1}}
}

\renewcommand{\symbl}[3]{\newglossaryentry{#1}{name ={#2},	description ={#3}}
  \expandafter\newcommand\csname #1\endcsname{\gls{#1}}
}


\newglossarystyle{dottedlocations}{%
   \glossarystyle{list}%
   \renewcommand*{\glossaryentryfield}[5]{%
   \item[\glsentryitem{##1}\glstarget{##1}{##2}] ##3, p %
       \unskip\leaders\hbox to 2.9mm{} ##5}%
   \renewcommand*{\glsgroupskip}{}%
}

 \newglossarystyle{acronyms}{%
 % put the glossary in the itemize environment:
 \renewenvironment{theglossary}%
   {\begin{tabbing}}{\end{tabbing}}%
 % have nothing after \begin{theglossary}:
 \renewcommand*{\glossaryheader}{}%
 % have nothing between glossary groups:
 \renewcommand*{\glsgroupheading}[1]{}%
 \renewcommand*{\glsgroupskip}{}%
 % set how each entry should appear:
 \renewcommand*{\glossentry}[2]{%
 \glstarget{##1}{\glossentryname{##1}}\\% \kill % the entry name
 \=\glossentrysymbol{##1}% the symbol in brackets
 \space \glossentrydesc{##1},% the description
 \space p. ##2\\% the number list in square brackets
 }%
 % set how sub-entries appear:
 \renewcommand*{\subglossentry}[3]{%
   \glossentry{##2}{##3}}%
}


 \newglossarystyle{symbols}{%
 % put the glossary in the itemize environment:
 \renewenvironment{theglossary}%
   {\begin{tabbing}}{\end{tabbing}}%
 % have nothing after \begin{theglossary}:
 \renewcommand*{\glossaryheader}{}%
 % have nothing between glossary groups:
 \renewcommand*{\glsgroupheading}[1]{}%
 \renewcommand*{\glsgroupskip}{}%
 % set how each entry should appear:
 \renewcommand*{\glossentry}[2]{%
 \glstarget{##1}{\glossentryname{##1}}\\% \kill % the entry name
 \=\textbf{\glossentrysymbol{##1}}% the symbol in brackets
 
 \space \glossentrydesc{##1},% the description
 \space p. ##2\\% the number list in square brackets
 }%
 % set how sub-entries appear:
 \renewcommand*{\subglossentry}[3]{%
   \glossentry{##2}{##3}}%
 }

\acr{UFRJ}{UFRJ}{Federal University of Rio de Janeiro}
\acr{LCA}{LCA}{Control and Automation Laboratory}

\acr{DAOCT}{DAOCT}{Deterministic Automaton
  with Outputs and Conditional Transitions}

\acr{DOF}{DOF}{Degrees of Freedom}
\acr{CCW}{CCW}{Counter Clockwise}
\acr{CW}{CW}{Clockwise}

\acr{HMI}{HMI}{Human-Machine Interface}
\acr{ECA}{ECA}{Engenharia de controle e Automação}
\acr{PLC}{PLC}{Programmable Logic Controller}
[PLCs][Programmable Logic Controllers]
\acr{DES}{DES}{Discrete Event System}
[DESs][Discrete Event Systems]
\acr{CSV}{CSV}{Comma Separated Values \texttt{.csv}}

\acr{LD}{LD}{Ladder Diagram}

\acr{CIPN}{CIPN}{Control Interpreted Petri Net}

% DAOCT
\symbl{OmegaSet}{$\Omega$}{$\Omega \subset \mathbb{N}_1^{m_i + m_0}$ Set of IO vectors}
\symbl{SigmaSet}{$\Sigma$}{Set of events}
\symbl{XSet}{$X$}{Set of states}
\symbl{ffunction}{$f$}{$f : X \times \Sigma^* \rightarrow X$ Deterministic
  transition function}
\symbl{lambdafunction}{$\lambda$}{$\lambda : X \rightarrow \Omega$ State
  output function}
\symbl{RSet}{$R$}{$R = \{1,2,\dots,r\} $ Set of path indices}
\symbl{thetafunction}{$\theta$}{$\theta : X \times \Sigma \rightarrow 2^R$ Path
  estimation function}
\symbl{xZero}{$x_0$}{Initial State}
\symbl{XfSet}{$X_f$}{$X_f \subseteq X$ Set of final states}
%%% Local Variables:
%%% mode: latex
%%% TeX-master: "./monografia.tex"
%%% End: