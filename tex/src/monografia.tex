\documentclass[masc,grad]{coppe} %Universal input encoding G.L.
% \documentclass[aprovado,masc,grad]{coppe} %Universal input encoding G.L.
\usepackage{graphicx}
\usepackage[utf8]{inputenc}
\usepackage[T1]{fontenc}
\usepackage{pgfpages}
\usepackage{multicol}
\usepackage{picture}
\usepackage{bbding}
% \usepackage[portuguese]{babel}
\uselanguage{Portuguese}
\languagepath{Portuguese}
\setbeamertemplate{caption}[numbered]
\usepackage{sansmathaccent}
\pdfmapfile{+Sensationsmache.map}
\graphicspath{{../../figures/}}
% \usepackage{movie15}
% \usepackage{media9}
\usepackage{stmaryrd} 
\usepackage{listings}
\definecolor{keywordstyle}{rgb}{0,0,0.82}
\definecolor{commentstyle}{rgb}{0,0.6,0}
\definecolor{numberstyle}{rgb}{0.5,0.5,0.5}
\definecolor{stringstyle}{rgb}{0.58,0,0.82}

% Listing options
\lstset{ 
  % backgroundcolor=\color{white},   % choose the background color; you must add \usepackage{color} or \usepackage{xcolor}; should come as last argument
  basicstyle=\footnotesize,        % the size of the fonts that are used for the code
  breakatwhitespace=false,         % sets if automatic breaks should only happen at whitespace
  breaklines=true,                 % sets automatic line breaking
  captionpos=t,                    % sets the caption-position to bottom
  commentstyle=\color{commentstyle},    % comment style
  deletekeywords={...},            % if you want to delete keywords from the given language
  escapeinside={\%*}{*)},          % if you want to add LaTeX within your code
  extendedchars=true,              % lets you use non-ASCII characters; for 8-bits encodings only, does not work with UTF-8
  % firstnumber=1000,                % start line enumeration with line 1000
  % frame=single,	                   % adds a frame around the code
  keepspaces=true,                 % keeps spaces in text, useful for keeping indentation of code (possibly needs columns=flexible)
  keywordstyle=\color{keywordstyle},       % keyword style
  % language=Octave,                 % the language of the code
  morekeywords={*,...},            % if you want to add more keywords to the set
  numbers=left,                    % where to put the line-numbers; possible values are (none, left, right)
  numbersep=10pt,                   % how far the line-numbers are from the code
  numberstyle=\tiny\color{numberstyle}, % the style that is used for the line-numbers
  rulecolor=\color{black},         % if not set, the frame-color may be changed on line-breaks within not-black text (e.g. comments (green here))
  showspaces=false,                % show spaces everywhere adding particular underscores; it overrides 'showstringspaces'
  showstringspaces=false,          % underline spaces within strings only
  showtabs=false,                  % show tabs within strings adding particular underscores
  stepnumber=2,                    % the step between two line-numbers. If it's 1, each line will be numbered
  stringstyle=\color{stringstyle},     % string literal style
  tabsize=2,	                   % sets default tabsize to 2 spaces
  title=\lstname                   % show the filename of files included with \lstinputlisting; also try caption instead of title
}

\usepackage{tikzscale}
\usepackage{tikz}
\usepackage{adjustbox}
\usepackage{pgfplots}
% \newcommand{\bulletpoint}[1]{\begin{itemize}
% 		\item #1
% \end{itemize}}

\newcommand{\bulletpoint}[1]{$\bullet$ #1}
\newif\ifdebug
\newcommand{\draft}{\debugtrue}
\newcommand{\final}{\debugfalse}
\newcommand{\includetikzfigure}[2][]{
    \ifdebug {\includegraphics[#1]{#2.pdf}}
    \else  \includegraphics[#1]{#2}\fi
}

\pgfplotsset{compat=newest}
\usepgfplotslibrary{groupplots}
\usepgfplotslibrary{dateplot}
% \newtheorem{theorem}{Teorema}
% \numberwithin{theorem}{section}

% \newtheorem{example}{Examplo}
% \numberwithin{example}{section}

% \newtheorem{definition}{Definição}
% \numberwithin{definition}{section}

% \newtheorem{observation}{Observação}
% \numberwithin{observation}{section}

\newlength{\ladderskip}
\newlength{\ladderrungsep}
\usetikzlibrary{patterns}
\usetikzlibrary{arrows,shapes,circuits.plc.ladder,external}
\usetikzlibrary{arrows,shapes,automata,petri,external,arrows.meta}
\tikzset{
	place/.style={
	circle,
	thick,
	% draw=black!100,
  % draw=blue!75,
    % fill=blue!20,
    minimum size=6mm
  },
  extPlace/.style={
    circle,
    dotted,
    % draw=black!100, % draw=blue!75,
    % fill=blue!20,
    minimum size=6mm
  },
  extTransition/.style={
    rectangle,
    dotted,
    fill=white,
    minimum width=8mm,
    inner ysep=0.7pt
  },
  transition/.style={
    rectangle,
    thick,
    fill=black,
    minimum width=8mm,
    inner ysep=0.7pt
  },
  extTimedtransition/.style={
    rectangle,
    dotted,
    fill=white,
    minimum width=8mm,
    inner ysep=2pt
  },
  timedtransition/.style={
    rectangle,
    thick,
    fill=white,
    minimum width=8mm,
    inner ysep=2pt
  },
  inhibitor/.style={-o},
  text/.style={}
}
\makeatletter
\tikz@def@grow@tokens{2}{1}{-1.}{0}
\tikz@def@grow@tokens{2}{2}{1.}{0}
% \tikz@def@grow@tokens{3}{1}{-1}{0}
% \tikz@def@grow@tokens{3}{2}{0}{1}
% \tikz@def@grow@tokens{3}{3}{1.5}{-1}
\makeatother
\newcommand{\colvec}[2][1]{%
  \scalebox{#1}{%
    \renewcommand{\arraystretch}{.7}%
    $\begin{bmatrix}#2\end{bmatrix}$%
  }
}












%%% Local Variables:
%%% mode: latex
%%% TeX-master: "presentation"
%%% End:


\newcommandx\acr[5][4=,5=]{
  \ifthenelse{\equal{#5}{}}
  {
    \acrSing{#1}{#2}{#3}
  }
  {
    \acrPl{#1}{#2}{#3}{#4}{#5}
  }
  } 

\newcommand{\acrSing}[3]{\newacronym{#1}{#2}{#3}
  \expandafter\newcommand\csname #1\endcsname{\gls{#1}}}

\newcommand{\acrPl}[5]{
  \newacronym[plural=#4,firstplural=#5 (#4)]{#1}{#2}{#3}
  \expandafter\newcommand\csname #1\endcsname{\gls{#1}}
  \expandafter\newcommand\csname #4\endcsname{\glspl{#1}}
}

\renewcommand{\symbl}[3]{\newglossaryentry{#1}{name ={#2},	description ={#3}}
  \expandafter\newcommand\csname #1\endcsname{\gls{#1}}
}


\newglossarystyle{dottedlocations}{%
   \glossarystyle{list}%
   \renewcommand*{\glossaryentryfield}[5]{%
   \item[\glsentryitem{##1}\glstarget{##1}{##2}] ##3, p %
       \unskip\leaders\hbox to 2.9mm{} ##5}%
   \renewcommand*{\glsgroupskip}{}%
}

 \newglossarystyle{acronyms}{%
 % put the glossary in the itemize environment:
 \renewenvironment{theglossary}%
   {\begin{tabbing}}{\end{tabbing}}%
 % have nothing after \begin{theglossary}:
 \renewcommand*{\glossaryheader}{}%
 % have nothing between glossary groups:
 \renewcommand*{\glsgroupheading}[1]{}%
 \renewcommand*{\glsgroupskip}{}%
 % set how each entry should appear:
 \renewcommand*{\glossentry}[2]{%
 \glstarget{##1}{\glossentryname{##1}}\\% \kill % the entry name
 \=\glossentrysymbol{##1}% the symbol in brackets
 \space \glossentrydesc{##1},% the description
 \space p. ##2\\% the number list in square brackets
 }%
 % set how sub-entries appear:
 \renewcommand*{\subglossentry}[3]{%
   \glossentry{##2}{##3}}%
}


 \newglossarystyle{symbols}{%
 % put the glossary in the itemize environment:
 \renewenvironment{theglossary}%
   {\begin{tabbing}}{\end{tabbing}}%
 % have nothing after \begin{theglossary}:
 \renewcommand*{\glossaryheader}{}%
 % have nothing between glossary groups:
 \renewcommand*{\glsgroupheading}[1]{}%
 \renewcommand*{\glsgroupskip}{}%
 % set how each entry should appear:
 \renewcommand*{\glossentry}[2]{%
 \glstarget{##1}{\glossentryname{##1}}\\% \kill % the entry name
 \=\textbf{\glossentrysymbol{##1}}% the symbol in brackets
 
 \space \glossentrydesc{##1},% the description
 \space p. ##2\\% the number list in square brackets
 }%
 % set how sub-entries appear:
 \renewcommand*{\subglossentry}[3]{%
   \glossentry{##2}{##3}}%
 }

\acr{UFRJ}{UFRJ}{Federal University of Rio de Janeiro}
\acr{LCA}{LCA}{Control and Automation Laboratory}

\acr{DAOCT}{DAOCT}{Deterministic Automaton
  with Outputs and Conditional Transitions}
\acr{ECA}{ECA}{Engenharia de controle e Automação}
\acr{PLC}{PLC}{Programmable Logic Controller}
[PLCs][Programmable Logic Controllers]
\acr{DES}{DES}{Discrete Event System}
[DESs][Discrete Event Systems]


\acr{LD}{LD}{Ladder Diagram}

\acr{CIPN}{CIPN}{Control Interpreted Petri Net}

% DAOCT
\symbl{OmegaSet}{$\Omega$}{$\Omega \subset \mathbb{N}_1^{m_i + m_0}$ Set of IO vectors}
\symbl{SigmaSet}{$\Sigma$}{Set of events}
\symbl{XSet}{$X$}{Set of states}
\symbl{ffunction}{$f$}{$f : X \times \Sigma^* \rightarrow X$ Deterministic
  transition function}
\symbl{lambdafunction}{$\lambda$}{$\lambda : X \rightarrow \Omega$ State
  output function}
\symbl{RSet}{$R$}{$R = \{1,2,\dots,r\} $ Set of path indices}
\symbl{thetafunction}{$\theta$}{$\theta : X \times \Sigma \rightarrow 2^R$ Path
  estimation function}
\symbl{xZero}{$x_0$}{Initial State}
\symbl{XfSet}{$X_f$}{$X_f \subseteq X$ Set of final states}
%%% Local Variables:
%%% mode: latex
%%% TeX-master: "./monografia.tex"
%%% End:
\makeindex
\makelosymbols
\makeloabbreviations
\acr{ECA}{ECA}{Engenharia de controle e Automação}
% \final
\draft

% include only to speed up tests
% \includeonly{test/examples}
\signedFrontPage{../../figures/poli-logo.pdf}

\begin{document}
% Identificação e Detecção de Falhas em um Sistema de Manufatura Didático
\title{Identification and Failure~Detection in a Didactic~Manufacture~System}
\foreigntitle{Identificação e Detecção~de~Falhas em um Sistema~de~Manufatura~Didático}
\author{Rafael Accácio}{Nogueira}

\advisor{Prof.}{Marcos Vicente de Brito}{Moreira}{D.Sc.}


% Case advisor is a woman add {f} Example:
% \advisor{Prof.}{Marie}{Sklodowksca-Curie}{PhD.}{f}

% \advisor{Prof.}{Nome do Segundo Orientador}{Sobrenome}{Ph.D.}
% \advisor{Prof.}{Nome do Terceiro Orientador}{Sobrenome}{D.Sc.}

% \examiner{Prof.}{Nome do Primeiro Examinador Sobrenome}{D.Sc.}
% \examiner{Prof.}{Nome do Segundo Examinador Sobrenome}{Ph.D.}
% \examiner{Prof.}{Nome do Terceiro Examinador Sobrenome}{D.Sc.}
\department{ECA}

\date{04}{2019}

\keyword{Failure~Detection}
\keyword{Discrete~Event~Systems}
\maketitle

\frontmatter


\makefrontpage\warning{TURN DEBUG OFF}\newpage


\makecatalog
\dedication{“It's a dangerous business going out your door. You step onto the road,
and if you don't keep your feet, there's no knowing where you might be swept off
to.” \\(J.R.R Tolkien)}
% ``É um negócio perigoso, Frodo, sair da sua porta. Você pisa na estrada, e, se
% não controlar seus pés, não há como saber até onde você pode ser levado''
% ``Se enxerguei mais longe, foi porque me apoiei sobre os ombros de gigantes.'' (Isaac Newton)

%%% Local Variables:
%%% mode: latex
%%% TeX-master: "../monografia.tex"
%%% End:

\include{frontMatter/thanks}

\begin{abstract}

Este trabalho tem como objetivo propor ferramentas e uma metodologia
para a identificação de sistemas a eventos
discretos, utilizando o modelo \gls{DAOCT}, que poderá ser usado para detecção
de falhas.
Para tanto, será realizado
o projeto de controle de um sistema de manufatura didático, utilizando
em uma primeira fase redes de petri, depois convertendo na linguagem
Ladder. Uma vez implementado o controle os dados de entrada e saída da planta
serão registrados e depois dados como entrada para o algoritmo de identificação do modelo
\gls{DAOCT}.
Cada um desses passos serão descritos nesse trabalho e o modelo identificado
será discutido.
\end{abstract}


%%% Local Variables:
%%% mode: latex
%%% TeX-master: "../monografia.tex"
%%% End:
\begin{foreignabstract}

This work has as primary objective to propose tools and a methodology
for identification and failure detection on discrete events systems
using the \gls{DAOCT} model. In order to accomplish this, the control
of a didactic manufacture system will be designed, using petri nets in
a first phase converting it into Ladder. Once the control is
implemented, it will be showed how to make the input and output data
acquisition necessary to feed the \gls{DAOCT} model identification
algorithm. The \gls{DAOCT} model identified by the offline program,
using data acquired when the system was operational in normal
conditions, will be used online to detect failures in tests where the
failures will be created by fiddling around with the sensors and
actuators, this way the model will be tested using relatively larger systems.

% Este trabalho tem como objetivo propor ferramentas e uma metodologia
% para a identificação e detecção de falhas em sistemas a eventos
% discretos, utilizando o modelo \gls{DAOCT}. Para tanto, será realizado
% o projeto de controle de um sistema de manufatura didático, utilizando
% em uma primeira fase redes de petri, depois convertendo na linguagem
% Ladder. Uma vez implementado o controle será mostrado como fazer a
% aquisição dos dados de entrada e saída da planta, necessários para o
% algoritmo de identificação do modelo \gls{DAOCT}. O modelo \gls{DAOCT}
% identificado pelo programa offline, usando dados colhidos em diversos testes no qual
% a planta se comporta normalmente, será usado para detectar falhas
% online em testes onde situações de falhas serão causadas ao
% alterar o comportamento de sensores e atuadores, assim testando o
% modelo para sistemas de relativamente maiores dimensões 


\end{foreignabstract}

%%% Local Variables:
%%% mode: latex
%%% TeX-master: "../monografia.tex"
%%% End:

\tableofcontents

\listoffigures
\listoftables
\printlosymbols
\printloabbreviations
\setglossarystyle{long4col}
\renewcommand*{\glsnamefont}[1]{\textmd{#1}}

\newglossarystyle{dottedlocations}{%
   \glossarystyle{list}%
   \renewcommand*{\glossaryentryfield}[5]{%
   \item[\glsentryitem{##1}\glstarget{##1}{##2}] ##3, p %
       \unskip\leaders\hbox to 2.9mm{} ##5}%
   \renewcommand*{\glsgroupskip}{}%
}

 \newglossarystyle{tabular}{%
 % put the glossary in the itemize environment:
 \renewenvironment{theglossary}%
   {\begin{tabular}{ll}}{\end{tabular}}%
 % have nothing after \begin{theglossary}:
 \renewcommand*{\glossaryheader}{}%
 % have nothing between glossary groups:
 \renewcommand*{\glsgroupheading}[1]{}%
 \renewcommand*{\glsgroupskip}{}%
 % set how each entry should appear:
 \renewcommand*{\glossentry}[2]{%

 \glstarget{##1}{\glossentryname{##1}}&% the entry name
  \glossentrysymbol{##1}% the symbol in brackets
 \space \glossentrydesc{##1},% the description
 \space p. ##2\\% the number list in square brackets
 }%
 % set how sub-entries appear:
 \renewcommand*{\subglossentry}[3]{%
   \glossentry{##2}{##3}}%
}

 \newglossarystyle{mylist}{%
 % put the glossary in the itemize environment:
 \renewenvironment{theglossary}%
   {\begin{tabbing}}{\end{tabbing}}%
 % have nothing after \begin{theglossary}:
 \renewcommand*{\glossaryheader}{}%
 % have nothing between glossary groups:
 \renewcommand*{\glsgroupheading}[1]{}%
 \renewcommand*{\glsgroupskip}{}%
 % set how each entry should appear:
 \renewcommand*{\glossentry}[2]{%

 \glstarget{##1}{\glossentryname{##1}}\\% \kill % the entry name
 \=\glossentrysymbol{##1}% the symbol in brackets
 \space \glossentrydesc{##1},% the description
 \space p. ##2\\% the number list in square brackets
 }%
 % set how sub-entries appear:
 \renewcommand*{\subglossentry}[3]{%
   \glossentry{##2}{##3}}%
 }

\glossarystyle{mylist}

\newacronym{manet}{MANET}{Mobile Ad hoc NETwork}
\newacronym{rwp}{RWP}{Random WayPoint}
\newacronym{rw}{RW}{Random Walk}
\glsaddall

\printglossary[type=\acronymtype]\newpage
\printglossary[]

\newpage

\mainmatter


\chapter{Examples}
\label{chap:examples}

\section{teste}
\subsection{teste}
\subsubsection{teste}
\begin{algorithm2e}
\caption{Gauss-Seidel Algorithm}\label{alg:gauss-seidel}
\KwIn
{%
scalar $\epsilon$,
matrix $\mathbf{A} = (a_{ij})$,
vector $\vec{b}$
and initial vector $\vec{x}^{(0)}$
}
\For{$k\leftarrow 1$ \KwTo maximum iterations}
{
\For{$i\leftarrow 1$ \KwTo $n$}
{
$
x_i^{(k)} =
\frac
{
b_i-\sum_{j=1}^{i-1}a_{ij}x_j^{(k)}
-\sum_{j=i+1}^{n}a_{ij}x_j^{(k-1)}
}%
{a_{ii}}
$\;
}
\If{$\lvert\vec{x}^{(k)}-\vec{x}^{(k-1)}\rvert < \epsilon$}
{Stop}
}
\end{algorithm2e}

% \begin{table}[H]
%   \centering
%   \caption{table}
%   \begin{tabular}{cc}
%     \label{tab:tab1}
%     \hypertarget{tab:1}{}
%     Transição&Significado\\
%     \hline \\
%     \hyperlink{partialNet:t1}{\hypertarget{partialTable:t1}{$t_{1}$}}&Test\\
%     \hyperlink{partialNet:p1}{\hypertarget{partialTable:p1}{$p_{1}$}}&balbalbal\\
%     \hyperlink{partialNet:p0m2}{\hypertarget{partialTable:p0m2}{$p_{0}$}}&balbalbal
%   \end{tabular}
% \end{table}

% \newpage
% \begin{figure}[h]
%   \centering
%   \input{../../figures/petriNets/example/test}
%   \caption{example }
%   \label{fig:example}
% \end{figure}

% \newpage
% \begin{figure}[h]
%   \centering
%   \begin{tikzpicture}[>=latex',line join=bevel,]
%%
\node (et1) at (27.0bp,18.0bp) [draw,ellipse,extTransition, label=above:, label=left:\hyperlink{partialNet:t1}{$t_{1}$},rotate=90] {};
  \node (p1) at (95.0bp,18.0bp) [draw,ellipse,place, label=above:, label=left:\hyperlink{partialTable:p1}{\hypertarget{partialNet:p1}{$p_{1}$}},rotate=90] {};
  \draw [-Latex] (et1) ..controls (54.266bp,18.0bp) and (57.727bp,18.0bp)  .. (p1);
%
\end{tikzpicture}

%   \caption{example }
%   \label{fig:example}
% \end{figure}

% \newpage

% \begin{figure}[H]
%   \centering
%   \includegraphics{../../figures/petriNets/dot/2-metalv/metalv.pdf}
%   \caption{qlksdjf}
%   \label{fig:example}
% \end{figure}


\OmegaSet

\begin{figure}[H]
  \centering
  \includegraphics[width=0.4\textwidth]{../../figures/tests/teste.tikz}
  \caption{petri net example}
  \label{fig:petrinetexample}
\end{figure}


\clearpage

\addtikzfigure{../../figures/petriNets/dot/1-initialization/initial}
{Petri net of Initialization module.}
{petriinitialization}

\begin{table}[htbp]
\caption{Lugares do Módulo de Inicialização}
\centering
\begin{tabular}{c|c}
Places & Meaning\\
\hline
P\textsubscript{0} & Sistema Parado\\
P\textsubscript{1} & Retrair Pistão MAG 1*\\
P\textsubscript{2} & Pistão MAG 1 Recuado\\
P\textsubscript{3} & Retrair Pistão MAG 2*\\
P\textsubscript{4} & Pistão MAG 2 Recuado\\
P\textsubscript{5} & Retrair pistão de descarte D*\\
P\textsubscript{6} & Pistão de descarte D Recuado\\
P\textsubscript{7} & Retrair pistão de descarte C*\\
P\textsubscript{8} & Pistão de descarte C Recuado\\
P\textsubscript{9} & Retrair pistão de descarte E*\\
P\textsubscript{10} & Pistão de descarte E Recuado\\
P\textsubscript{11} & Ligar esteira sentido reverso\\
P\textsubscript{12} & Esteira limpa\\
P\textsubscript{13} & Resetar Variáveis\footnotemark\\
P\textsubscript{14} & Retrair atuador vertical da prensa\\
P\textsubscript{15} & Levantar porta da prensa\\
P\textsubscript{16} & Estender atuador horizontal da prensa\\
P\textsubscript{17} & Prensa pronta\\
P\textsubscript{18} & Braço retraído e armazenador de cubos retraído na horizontal\\
P\textsubscript{19} & Mover armazenador para direita\\
P\textsubscript{20} & Armazenador pronto na horizontal\\
P\textsubscript{21} & Mover armazenador para baixo\\
P\textsubscript{22} & Armazenador pronto na vertical\\
P\textsubscript{23} & Girar braço sentido antihorário\footnotemark\\
P\textsubscript{24} & Braço parado\\
P\textsubscript{25} & Girar braço sentido horário\textsuperscript{\ref{org871b750}} e Habilita HSC\\
P\textsubscript{26} & Braço parado frente a esteira\\
P\textsubscript{27} & Sistema Pronto\\
\end{tabular}
\end{table}\footnotetext[1]{\label{org8181e89}Variáveis IEC\textsubscript{COUNTER}, IEC\textsubscript{COUNTER1}, IEC\textsubscript{COUNTER2}, IEC\textsubscript{COUNTER3}, IEC\textsubscript{COUNTER4}, IEC\textsubscript{COUNTER5}.}\footnotetext[2]{\label{org871b750}Verificar sentido de rotação do braço.}

\begin{table}[htbp]
\caption{Transições do Módulo de Inicialização}
\centering
\begin{tabular}{ll}
Transitions & Meaning\\
t\textsubscript{0} & Botão de inicialização\\
t\textsubscript{1} & Sensor MAG 1 Retraído\\
t\textsubscript{2} & Sensor MAG 2 Retraído\\
t\textsubscript{3} & Sensor pistão de descarte D Retraído\\
t\textsubscript{4} & Sensor pistão de descarte C Retraído\\
t\textsubscript{5} & Sensor pistão de descarte E Retraído\\
t\textsubscript{6} & \\
t\textsubscript{7} & T=15s\\
t\textsubscript{8} & T=2.5s\\
t\textsubscript{9} & Sensor porta prensa aberta\\
t\textsubscript{10} & Sensor atuador horizontal da prensa estendido\\
t\textsubscript{11} & Sensor Hz armazenador de cubos e braço retraídos\\
t\textsubscript{12} & Fim de curso direito do armazenador de cubos\\
t\textsubscript{13} & Fim de curso inferior do armazenador de cubos\\
t\textsubscript{14} & T=2s\\
t\textsubscript{15} & Sensor Indutivo do braço\\
t\textsubscript{16} & T=1s\\
t\textsubscript{17} & Count\_300C.DB.Countval = \todo{-1690}\\
t\textsubscript{18} & \\
t\textsubscript{19} & Botão Começar\\
\end{tabular}
\end{table}


\addtikzfigure{../../figures/petriNets/dot/2-metalv/metalv}
{Petri net of metal cube half sorting module.}
{petri_initialization}

\begin{table}[htbp]
\caption{Lugares do Módulo 2 pt 1}
\centering
\begin{tabular}{ll}
p\(_{\text{28}}\) & Mag1 vazio\\
p\(_{\text{29}}\) & Mag1 com peça\\
p\(_{\text{30}}\) & Estender Mag1 Horizontal*\\
p\(_{\text{31}}\) & Retrair Mag1 Horizontal*\\
p\(_{\text{32}}\) & Mag1 Horizontal retraído\\
p\(_{\text{33}}\) & Ligar esteira sentido normal\\
p\(_{\text{34}}\) & \\
p\(_{\text{35}}\) & Peça de Plástico\\
p\(_{\text{36}}\) & Ligar esteira sentido normal\\
p\(_{\text{37}}\) & Estender Pistão de descarte D*\\
p\(_{\text{38}}\) & Retrair Pistão de descarte D*\\
p\(_{\text{39}}\) & Ligar esteira sentido normal\\
p\(_{\text{40}}\) & Estender Pistão de descarte C*\\
p\(_{\text{41}}\) & Retrair Pistão de descarte C*\\
p\(_{\text{42}}\) & \\
p\(_{\text{43}}\) & Peça de Metal\\
p\(_{\text{44}}\) & Ligar esteira sentido normal\\
p\(_{\text{45}}\) & Estender Pistão de descarte E*\\
p\(_{\text{46}}\) & Retrair Pistão de descarte E*\\
p\(_{\text{47}}\) & Ligar esteira sentido normal\\
p\(_{\text{48}}\) & Ligar esteira sentido normal\\
p\(_{\text{49}}\) & Peça Metal Pronta\\
p\(_{\text{50}}\) & Esteira Parada\\
\end{tabular}
\end{table}

\begin{table}[htbp]
\caption{Transições do Módulo 2 pt 1}
\centering
\begin{tabular}{ll}
t\(_{\text{20}}\) & \(\overline{\mbox{Sensor Chave de Presença de Peça Mag1}}\)\\
t\(_{\text{21}}\) & \\
t\(_{\text{22}}\) & Mag1 Horizontal estendido \(\uparrow\)\\
t\(_{\text{23}}\) & Mag1 Horizontal retraído \(\uparrow\)\\
t\(_{\text{24}}\) & T=0.5s\\
t\(_{\text{25}}\) & Presença \(\uparrow\) T=0.5s\\
t\(_{\text{26}}\) & \(\overline{\mbox{Sensor Metal}}\)\\
t\(_{\text{27}}\) & Sensor Preto\\
t\(_{\text{28}}\) & Presença Pistão de D \(\uparrow\)\\
t\(_{\text{29}}\) & Sensor pistão de descarte D estendido\\
t\(_{\text{30}}\) & Sensor pistão de descarte D retraído\\
t\(_{\text{31}}\) & Sensor Branco\\
t\(_{\text{32}}\) & Presença Pistão de C \(\uparrow\)\\
t\(_{\text{33}}\) & Sensor pistão de descarte C estendido\\
t\(_{\text{34}}\) & Sensor pistão de descarte C retraído\\
t\(_{\text{35}}\) & Sensor Metal\\
t\(_{\text{36}}\) & Sensor peça concavidade para baixo\\
t\(_{\text{37}}\) & Presença Pistão de E \(\uparrow\)\\
t\(_{\text{38}}\) & Sensor pistão de descarte E estendido\\
t\(_{\text{39}}\) & Sensor pistão de descarte E retraído\\
t\(_{\text{40}}\) & \\
t\(_{\text{41}}\) & Sensor peça concavidade para cima\\
t\(_{\text{42}}\) & Sensor final da esteira \(\uparrow\)\\
t\(_{\text{43}}\) & T=0.5s\\
t\(_{\text{44}}\) & Sensor final da esteira \(\downarrow\)\\
t\(_{\text{45}}\) & \\
\end{tabular}
\end{table}


\addtikzfigure{../../figures/petriNets/dot/3-plastic^/plastic}
{Petri net of plastic cube half sorting module.}
{petri_initialization}

\begin{table}[htbp]
\caption{Lugares do Módulo 2 pt 2}
\centering
\begin{tabular}{ll}
p\(_{\text{51}}\) & Mag2 vazio\\
p\(_{\text{52}}\) & Mag2 com peça\\
p\(_{\text{53}}\) & Estender Mag2 Horizontal*\\
p\(_{\text{54}}\) & Retrair Mag2 Horizontal*\\
p\(_{\text{55}}\) & Mag2 Horizontal Retraído\\
p\(_{\text{56}}\) & Ligar esteira sentido normal\\
p\(_{\text{57}}\) & \\
p\(_{\text{58}}\) & Ligar esteira sentido normal\\
p\(_{\text{59}}\) & Estender Pistão de descarte E*\\
p\(_{\text{60}}\) & Retrair Pistão de descarte E*\\
p\(_{\text{61}}\) & Peça de Metal\\
p\(_{\text{62}}\) & Ligar esteira sentido normal\\
p\(_{\text{63}}\) & Estender Pistão de descarte D*\\
p\(_{\text{64}}\) & Retrair Pistão de descarte D*\\
p\(_{\text{65}}\) & Peça Branca\\
p\(_{\text{66}}\) & Ligar esteira sentido normal\\
p\(_{\text{67}}\) & Estender Pistão de descarte C*\\
p\(_{\text{68}}\) & Retrair Pistão de descarte C*\\
p\(_{\text{69}}\) & \\
p\(_{\text{70}}\) & Ligar esteira sentido normal\\
p\(_{\text{71}}\) & Ligar esteira sentido normal\\
p\(_{\text{72}}\) & Peça branca pronta\\
p\(_{\text{73}}\) & Esteira parada\\
\end{tabular}
\end{table}

\begin{table}[htbp]
\caption{Transições do Módulo 2 pt 2}
\centering
\begin{tabular}{ll}
t\(_{\text{46}}\) & \(\overline{\mbox{Sensor Chave de Presença de Peça Mag2}}\)\\
t\(_{\text{47}}\) & \\
t\(_{\text{48}}\) & Mag2 Horizontal estendido \(\uparrow\)\\
t\(_{\text{49}}\) & Mag2 Horizontal retraído \(\uparrow\)\\
t\(_{\text{50}}\) & T=0.5s\\
t\(_{\text{51}}\) & Presença \(\uparrow\) T=0.5s\\
t\(_{\text{52}}\) & Sensor Metal\\
t\(_{\text{53}}\) & Presença Pistão de E \(\uparrow\)\\
t\(_{\text{54}}\) & Sensor pistão de descarte E estendido\\
t\(_{\text{55}}\) & Sensor pistão de descarte E retraído\\
t\(_{\text{56}}\) & \(\overline{\mbox{Sensor Metal}}\)\\
t\(_{\text{57}}\) & Sensor Preto\\
t\(_{\text{58}}\) & Presença Pistão de D \(\uparrow\)\\
t\(_{\text{59}}\) & Sensor pistão de descarte D estendido\\
t\(_{\text{60}}\) & Sensor pistão de descarte D retraído\\
t\(_{\text{61}}\) & Sensor Branco\\
t\(_{\text{62}}\) & Sensor peça concavidade para cima\\
t\(_{\text{63}}\) & Presença Pistão de C \(\uparrow\)\\
t\(_{\text{64}}\) & Sensor pistão de descarte C estendido\\
t\(_{\text{65}}\) & Sensor pistão de descarte C retraído\\
t\(_{\text{66}}\) & \\
t\(_{\text{67}}\) & Sensor peça concavidade para baixo\\
t\(_{\text{68}}\) & Sensor final da esteira \(\uparrow\)\\
t\(_{\text{69}}\) & T=0.5s\\
t\(_{\text{70}}\) & Sensor final da esteira \(\downarrow\)\\
t\(_{\text{71}}\) & \\
\end{tabular}
\end{table}


\addtikzfigure{../../figures/petriNets/dot/4-armBeltToPress/armBeltToPress}
{Petri net of manipulator taking a cube half from conveyor belt to assembly unit
  module.}
{petri_initialization}

\begin{table}[htbp]
\caption{Lugares do Módulo Braço Esteira Prensa}
\centering
\begin{tabular}{ll}
p\textsubscript{74} & Estender verticalmente o braço\\
p\textsubscript{75} & Estender vertical e horizontalmente o braço e Ligar o vácuo\\
p\textsubscript{76} & Estender horizontalmente o braço e Ligar o vácuo\\
p\textsubscript{77} & Estender vertical e horizontalmente o braço e Ligar o vácuo\\
p\textsubscript{78} & Estender verticalmente o braço e Ligar o vácuo\\
p\textsubscript{79} & Habilita HSC e Estender verticalmente o braço, Ligar o vácuo e Girar Braço no sentido horário\\
p\textsubscript{80} & Estender vertical e horizontalmente o braço e Ligar o vácuo\\
p\textsubscript{81} & Estender horizontalmente o braço e Ligar o vácuo\\
p\textsubscript{82} & Estender horizontalmente o braço\\
p\textsubscript{83} & Estender vertical e horizontalmente o braço\\
p\textsubscript{84} & Estender verticalmente o braço\\
p\textsubscript{85} & Habilita HSC e Estender verticalmente o braço e Girar Braço no sentido antihorário\\
p\textsubscript{86} & Estender Verticalmente o braço e IEC\textsubscript{COUNTER}:=IEC\textsubscript{COUNTER}+1\\
\end{tabular}
\end{table}

\begin{table}[htbp]
\caption{Transições do Módulo Braço Esteira Prensa}
\centering
\begin{tabular}{ll}
t\textsubscript{72} & Sensor vertical braço estendido\\
t\textsubscript{73} & T=1.5s\\
t\textsubscript{74} & T=1.5s e Sensor vertical braço retraído\\
t\textsubscript{75} & T=1.5s e Sensor vertical braço estendido\\
t\textsubscript{76} & T=1.5s e Sensor vertical braço estendido\\
t\textsubscript{77} & Count\_300C.DB.CountVal = \todo{-3330}\\
t\textsubscript{78} & T=1.5s e Sensor vertical braço estendido\\
t\textsubscript{79} & T=1.5s e Sensor vertical braço retraído\\
t\textsubscript{80} & T=1.5s\\
t\textsubscript{81} & T=1.5s e Sensor vertical braço estendido\\
t\textsubscript{82} & IEC\textsubscript{COUNTER0.DB}=1 e Sensor Hz prensa estendido e porta prensa aberta\\
t\textsubscript{83} & T=1.5s e IEC\textsubscript{COUNTER0.DB}=0 e Sensor vertical braço estendido\\
t\textsubscript{84} & Count\_300C.DB.CountVal = \todo{-1690}\\
t\textsubscript{85} & \\
\end{tabular}
\end{table}

 
\addtikzfigure{../../figures/petriNets/dot/5-press/press}
{Petri net of assembly unit module.}
{petri_initialization}

\begin{table}[htbp]
\caption{Lugares do Módulo prensa cubo}
\centering
\begin{tabular}{c|c}
Places & Meaning\\
\hline
\hyperlink{partialNet:p87}{\hypertarget{partialTable:p87}{$p_{87}$}} & Retrair atuador horizontal prensa*\\
\hyperlink{partialNet:p88}{\hypertarget{partialTable:p88}{$p_{88}$}} & Fechar Porta prensa*\\
\hyperlink{partialNet:p89}{\hypertarget{partialTable:p89}{$p_{89}$}} & Estender atuador vertical prensa*\\
\hyperlink{partialNet:p90}{\hypertarget{partialTable:p90}{$p_{90}$}} & Retrair atuador vertical prensa*\\
\hyperlink{partialNet:p91}{\hypertarget{partialTable:p91}{$p_{91}$}} & Abrir Porta prensa*\\
\hyperlink{partialNet:p92}{\hypertarget{partialTable:p92}{$p_{92}$}} & Estender atuador horizontal prensa*\\
\hyperlink{partialNet:p93}{\hypertarget{partialTable:p93}{$p_{93}$}} & Cubo pronto\\
\hyperlink{partialNet:p94}{\hypertarget{partialTable:p94}{$p_{94}$}} & Estender horizontalmente o braço e Ligar Vácuo\\
\hyperlink{partialNet:p95}{\hypertarget{partialTable:p95}{$p_{95}$}} & Estender verticalmente o braço\\
 & \\
\end{tabular}
\end{table}

\begin{center}
\begin{tabular}{c|c}
Transitions & Meaning\\
\hline
\hyperlink{partialNet:tt86}{\hypertarget{partialTable:tt86}{$t_{86}$}} & T=1s e Sensosr horizontal prensa retraído\\
\hyperlink{partialNet:tt87}{\hypertarget{partialTable:tt87}{$t_{87}$}} & T=1s e Sensor porta fechada\\
\hyperlink{partialNet:tt88}{\hypertarget{partialTable:tt88}{$t_{88}$}} & T=1s\\
\hyperlink{partialNet:tt89}{\hypertarget{partialTable:tt89}{$t_{89}$}} & T=1s\\
\hyperlink{partialNet:tt90}{\hypertarget{partialTable:tt90}{$t_{90}$}} & T=1s e sensor porta aberta\\
\hyperlink{partialNet:tt91}{\hypertarget{partialTable:tt91}{$t_{91}$}} & T=1s e sensor horizontal prensa estendido\\
\hyperlink{partialNet:t92}{\hypertarget{partialTable:t92}{$t_{92}$}} & \\
\hyperlink{partialNet:tt93}{\hypertarget{partialTable:tt93}{$t_{93}$}} & T=1.5s e Sensor horizontal do braço estendido\\
\end{tabular}
\end{center}


\addtikzfigure{../../figures/petriNets/dot/6-armPressToStorage/armPressToStorage}
{Petri net of manipulator taking cube from assembly unit to storage module.}
{petri_initialization}

\begin{table}[htbp]
\caption{Lugares do Módulo braço prensa armazenador}
\centering
\begin{tabular}{ll}
p\(_{\text{96}}\) & Estender horizontalmente o braço e Ligar Vácuo\\
p\(_{\text{97}}\) & Estender vertical e horizontalmente o braço e Ligar Vácuo\\
p\(_{\text{98}}\) & Resetar IEC\(_{\text{COUNTER0}}\)*, estender verticalmente o braço e Ligar Vácuo\\
p\(_{\text{99}}\) & Habilita HSC e Estender verticalmente o braço, Ligar Vácuo e Girar o Braço no sentido horário\\
p\(_{\text{100}}\) & Estender vertical e horizontalmente o braço e Ligar Vácuo\\
p\(_{\text{101}}\) & Estender horizontalmente o braço e Ligar Vácuo\\
p\(_{\text{102}}\) & Estender horizontalmente o braço\\
p\(_{\text{103}}\) & Estender vertical e horizontalmente o braço\\
p\(_{\text{104}}\) & Girar o braço no sentido antihorário\\
p\(_{\text{105}}\) & Braço parado\\
p\(_{\text{106}}\) & Habilita HSC e Girar o braço no sentido horário\\
p\(_{\text{107}}\) & Braço na esteira\\
\end{tabular}
\end{table}

\begin{table}[htbp]
\caption{Transições do Módulo braço prensa armazenador}
\centering
\begin{tabular}{ll}
t\(_{\text{94}}\) & T=1.5s e Sensor vertical braço retraído\\
t\(_{\text{95}}\) & Sensor vertical braço estendido, Fim de curso inferior e direito armazenador\\
t\(_{\text{96}}\) & \\
t\(_{\text{97}}\) & Count\_300C.DB.CountVal = \todo{-4920}\\
t\(_{\text{98}}\) & T=2s\\
t\(_{\text{99}}\) & T=2s\\
t\(_{\text{100}}\) & Sensor vertical braço retraído\\
t\(_{\text{101}}\) & Sensor vertical braço estendido, Fim de curso inferior e direito armazenador\\
t\(_{\text{102}}\) & Sensor indutivo do braço\\
t\(_{\text{103}}\) & T=1s\\
t\(_{\text{104}}\) & Count\_300C.DB.CountVal = \todo{-1690}\\
\end{tabular}
\end{table}


\addtikzfigure{../../figures/petriNets/dot/7-storageY/storageY}
{Petri net of storage unit positioning module (y axis).}
{petri_initialization}

\begin{table}[htbp]
\caption{Lugares do Módulo armazenador y}
\centering
\begin{tabular}{ll}
p\textsubscript{108} & cubo no armazenador\\
p\textsubscript{109} & mover armazenador para direita\\
p\textsubscript{110} & \\
p\textsubscript{111} & COUNTER3:=COUNTER3+1 mover armazenador para cima\\
p\textsubscript{112} & Reset COUNTER3*\\
p\textsubscript{113} & COUNTER3:=COUNTER3+1 mover armazenador para cima\\
p\textsubscript{114} & Reset COUNTER3*\\
p\textsubscript{115} & COUNTER3:=COUNTER3+1 mover armazenador para cima\\
p\textsubscript{116} & Reset COUNTER3*\\
p\textsubscript{117} & COUNTER3:=COUNTER3+1 mover armazenador para cima\\
p\textsubscript{118} & Reset COUNTER3*\\
p\textsubscript{119} & \\
\end{tabular}
\end{table}

\begin{table}[htbp]
\caption{Transições do Módulo armazenador y}
\centering
\begin{tabular}{ll}
t\textsubscript{105} & T=2s\\
t\textsubscript{106} & T=2s\\
t\textsubscript{107} & COUNTER2=0\\
t\textsubscript{108} & COUNTER3=4\\
t\textsubscript{109} & \\
t\textsubscript{110} & COUNTER2=1\\
t\textsubscript{111} & COUNTER3=3\\
t\textsubscript{112} & \\
t\textsubscript{113} & COUNTER2=2\\
t\textsubscript{114} & COUNTER3=2\\
t\textsubscript{115} & \\
t\textsubscript{116} & COUNTER2=3\\
t\textsubscript{117} & COUNTER3=1\\
t\textsubscript{118} & \\
\end{tabular}
\end{table}


\addtikzfigure{../../figures/petriNets/dot/8-storageX/storageX}
{Petri net of storage unit positioning module (x axis).}
{petri_initialization}

\begin{table}[htbp]
\caption{Lugares do Módulo armazenador (x)}
\centering
\begin{tabular}{ll}
p\textsubscript{120} & COUNTER1:=COUNTER1+1 e COUNTER4:=COUNTER4+1\\
p\textsubscript{121} & COUNTER5:=COUNTER5+1 e mover armazenador para a esquerda\\
p\textsubscript{122} & Reset COUNTER5*\\
p\textsubscript{123} & COUNTER5:=COUNTER5+1 e mover armazenador para a esquerda\\
p\textsubscript{124} & Reset COUNTER5*\\
p\textsubscript{125} & COUNTER5:=COUNTER5+1 e mover armazenador para a esquerda\\
p\textsubscript{126} & Reset COUNTER5*\\
p\textsubscript{127} & COUNTER5:=COUNTER5+1 e mover armazenador para a esquerda\\
p\textsubscript{128} & Reset COUNTER5*\\
p\textsubscript{129} & COUNTER5:=COUNTER5+1 e mover armazenador para a esquerda\\
p\textsubscript{130} & Reset COUNTER5*\\
p\textsubscript{131} & COUNTER5:=COUNTER5+1 e mover armazenador para a esquerda\\
p\textsubscript{132} & Reset COUNTER5*\\
p\textsubscript{133} & COUNTER5:=COUNTER5+1 e mover armazenador para a esquerda\\
p\textsubscript{134} & Reset COUNTER5* Reset COUNTER4* , COUNTER2:=COUNTER2+1\\
p\textsubscript{135} & \\
\end{tabular}
\end{table}

\begin{table}[htbp]
\caption{Transições Módulo armazenador (x)}
\centering
\begin{tabular}{ll}
t\textsubscript{119} & COUNTER4=0\\
t\textsubscript{120} & COUNTER5=1\\
t\textsubscript{121} & \\
t\textsubscript{122} & COUNTER4=1\\
t\textsubscript{123} & COUNTER5=2\\
t\textsubscript{124} & \\
t\textsubscript{125} & COUNTER4=2\\
t\textsubscript{126} & COUNTER5=3\\
t\textsubscript{127} & \\
t\textsubscript{128} & COUNTER4=3\\
t\textsubscript{129} & COUNTER5=4\\
t\textsubscript{130} & \\
t\textsubscript{131} & COUNTER4=4\\
t\textsubscript{132} & COUNTER5=5\\
t\textsubscript{133} & \\
t\textsubscript{134} & COUNTER4=5\\
t\textsubscript{135} & COUNTER5=8\\
t\textsubscript{136} & \\
t\textsubscript{137} & COUNTER4=6\\
t\textsubscript{138} & COUNTER5=9\\
t\textsubscript{139} & \\
\end{tabular}
\end{table}


\addtikzfigure{../../figures/petriNets/dot/9-storePiece/storePiece}
{Petri net of cube storage module.}
{petri_initialization}

\begin{table}[htbp]
\caption{Places from the cube storage module.}
\centering
\begin{tabular}{ll}
p\(_{\text{136}}\) & Estender horizontalmente armazenador\\
p\(_{\text{137}}\) & Estender horizontalmente armazenador e mover armazenador para baixo\\
p\(_{\text{138}}\) & Estender horizontalmente armazenador\\
p\(_{\text{139}}\) & Peça armazenada\\
p\(_{\text{140}}\) & Mover armazenador para a direita\\
p\(_{\text{141}}\) & Armazenador pronto na horizontal\\
p\(_{\text{142}}\) & Mover armazenador para baixo\\
p\(_{\text{143}}\) & Armazenador pronto na vertical\\
p\(_{\text{144}}\) & \\
p\(_{\text{145}}\) & Armazenador pronto\\
\end{tabular}
\end{table}

\begin{table}[htbp]
\caption{Transitions from the cube storage module.}
\centering
\begin{tabular}{ll}
t\(_{\text{140}}\) & T=2s\\
t\(_{\text{141}}\) & T=3s\\
t\(_{\text{142}}\) & T=0.25s\\
t\(_{\text{143}}\) & T=3s\\
t\(_{\text{144}}\) & T=7s\\
t\(_{\text{145}}\) & Fim de curso direito do armazenador\\
t\(_{\text{146}}\) & Fim de curso inferior do armazenador\\
t\(_{\text{147}}\) & \\
t\(_{\text{148}}\) & COUNTER1<28\\
t\(_{\text{149}}\) & COUNTER1=28\\
\end{tabular}
\end{table}







%%% Local Variables:
%%% mode: latex
%%% TeX-master: "../monografia"
%%% End:


\section{Introdução}


\begin{frame}{Motivação}
\begin{itemize}
\item Maior parte da produção de bens é industrializada \pause
\item Falhas no processo $\rightarrow$ Interrupção na produção \pause
\item Interrupção na produção = prejuízo \pause 
\item Solução: Deteção de falha \pause
\item \citep{davis1988model}: Conhecer como deveria funcionar para determinar
  porque não está.
  \note{Parafraseando DAVIS e HAMSCHER} 
\end{itemize}
\end{frame}

\begin{frame}{Objetivo}
\begin{itemize}
\item Modelar sistema a eventos discretos por identificação \\com detecção de falhas como fim.
  \note{objetivo desse trabalho é mostrar um método de modelagem por
    identificação.}\pause
\item Apresentar método da concepção do controle do sistema à identificação 
\end{itemize}
\end{frame}


%%% Local Variables:
%%% mode: latex
%%% TeX-master: "../presentation"
%%% End:
\chapter{Revisão Bibliográfica}
como foi visto no capítulo \ref{ch:teste}

%%% Local Variables:
%%% mode: latex
%%% TeX-master: "../main"
%%% End:

\include{mainMatter/chap03}
\include{mainMatter/chap04}
\include{mainMatter/chap05}

\gls{ECA}
Teste de teste
\begin{table}[H]
  \centering
  \begin{tabular}{cc}
    \label{tab:tab1}
    \hypertarget{tab:1}{}
    Transição&Significado\\
    \hline \\
    \hyperlink{net:1}{$t_{1}$}&Test

  \end{tabular}
  \caption{table}
\end{table}

\backmatter
\bibliographystyle{coppe-unsrt}
\nocite{*}
\bibliography{bibliography}

\appendix
\chapter{Algumas Demonstrações}




%%% Local Variables:
%%% mode: latex
%%% TeX-master: "../monografia.tex"
%%% End:
\end{document}
