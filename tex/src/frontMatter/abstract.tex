\begin{foreignabstract}

This work has as primary objective to propose tools and a methodology
for identification and failure detection on discrete events systems
using the \gls{DAOCT} model. In order to accomplish this, the control
of a didactic manufacture system will be designed, using petri nets in
a first phase converting it into Ladder. Once the control is
implemented, it will be showed how to make the input and output data
acquisition necessary to feed the \gls{DAOCT} model identification
algorithm. The \gls{DAOCT} model identified by the offline program,
using data acquired when the system was operational in normal
conditions, will be used online to detect failures in tests where the
failures will be created by fiddling around with the sensors and
actuators, this way the model will be tested using relatively larger systems.

% Este trabalho tem como objetivo propor ferramentas e uma metodologia
% para a identificação e detecção de falhas em sistemas a eventos
% discretos, utilizando o modelo \gls{DAOCT}. Para tanto, será realizado
% o projeto de controle de um sistema de manufatura didático, utilizando
% em uma primeira fase redes de petri, depois convertendo na linguagem
% Ladder. Uma vez implementado o controle será mostrado como fazer a
% aquisição dos dados de entrada e saída da planta, necessários para o
% algoritmo de identificação do modelo \gls{DAOCT}. O modelo \gls{DAOCT}
% identificado pelo programa offline, usando dados colhidos em diversos testes no qual
% a planta se comporta normalmente, será usado para detectar falhas
% online em testes onde situações de falhas serão causadas ao
% alterar o comportamento de sensores e atuadores, assim testando o
% modelo para sistemas de relativamente maiores dimensões 


\end{foreignabstract}

%%% Local Variables:
%%% mode: latex
%%% TeX-master: "../monografia.tex"
%%% End: