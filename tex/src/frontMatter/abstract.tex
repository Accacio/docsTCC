\begin{foreignabstract}

This work has as primary objective to propose tools and a methodology
for identification of discrete events systems
using the \gls{DAOCT} model, which can be used to failure detection.
In order to accomplish this, the control
of a didactic manufacture system will be designed, using petri nets in
a first phase and then converting it into Ladder logic. Once the control is
implemented, the inputs and outputs of the plant will be logged and then
fed to the \gls{DAOCT} model identification
algorithm. Each one of this steps will be depicted in this work and the identified
model will be discussed.

% Este trabalho tem como objetivo propor ferramentas e uma metodologia
% para a identificação e detecção de falhas em sistemas a eventos
% discretos, utilizando o modelo \gls{DAOCT}. Para tanto, será realizado
% o projeto de controle de um sistema de manufatura didático, utilizando
% em uma primeira fase redes de petri, depois convertendo na linguagem
% Ladder. Uma vez implementado o controle será mostrado como fazer a
% aquisição dos dados de entrada e saída da planta, necessários para o
% algoritmo de identificação do modelo \gls{DAOCT}. O modelo \gls{DAOCT}
% identificado pelo programa offline, usando dados colhidos em diversos testes no qual
% a planta se comporta normalmente, será usado para detectar falhas
% online em testes onde situações de falhas serão causadas ao
% alterar o comportamento de sensores e atuadores, assim testando o
% modelo para sistemas de relativamente maiores dimensões


\end{foreignabstract}

%%% Local Variables:
%%% mode: latex
%%% TeX-master: "../monografia.tex"
%%% End: