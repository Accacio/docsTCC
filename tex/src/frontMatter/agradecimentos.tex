Gostaria de agradecer primeiramente a Deus, pois sem Ele nada é
possível e por \textbf{todas} as pessoas qu'Ele colocou em meu
caminho, que me fizeram crescer e ser o indivíduo que hoje sou.

Agradeço aos meus pais, Rosemeri e Rogério. Por todo amor e carinho, pela
atenção dada e a primeira educação, pontapé inicial essencial para toda minha
trajetória, educação não só acadêmica, mas também moral. Agradeço também por
terem sempre escolhido as melhores escolas que proporcionaram o conhecimento
necessário para entrar no Colégio Pedro II. 

Gostaria de agradecer a todos meus professores e professoras por terem mostrado o quão importante e bonita é a
   profissão e por terem sempre instigado a sede pelo aprendizado.
   Agradeço a todos que contribuíram para minha base acadêmica e profissional.

Agradeço a todas as amizades que fiz, principalmente as do Pedro II, com os
   quais convivi durante 7 anos, passando o fim da infância e por grande parte
   da adolêscencia e também as da UFRJ, mais especificamente da nossa turma T17,
   pois se chegamos até onde chegamos foi porque estivemos juntos, fortes, lado a lado,
   ombro no ombro, não deixando o outro cair, mas quando alguém caía sempre uma
   mão se estendia para ajudar a levantar e recomeçar.
Dos amigos da T17 alguns não poderiam não ser mencionados:

Os melhores companheiros de grupo, Gabriel Pelielo e Rodrigo Moysés, um
verdadeiro ``Power Trio'', sinergia define bem todos trabalhos que fizemos. E também os amigos Philipe Moura e
Felipe Matheus, que me incentivaram a sair da minha zona de conforto e me
fizeram compreender de fato o sentido do quão ``perigoso'' é sair pela porta de casa, pois
quando saímos da nossa zona de conforto, coisas mágicas podem acontecer e
pessoas mágicas podem aparecer em nossas vidas.

%%% Local Variables:
%%% mode: latex
%%% TeX-master: "../monografia.tex"
%%% End:
