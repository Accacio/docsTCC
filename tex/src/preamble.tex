\maxdeadcycles=20
\usepackage[utf8]{inputenc}
\usepackage[english]{babel}
% \usepackage[brazil]{babel}
\usepackage[T1]{fontenc}
\usepackage{graphicx}
\usepackage{pax}
\usepackage{tikzscale}
\usepackage[titletoc]{appendix}
\usepackage{pgfplots}
\pgfplotsset{compat=newest}
\usepgfplotslibrary{groupplots}
\usepgfplotslibrary{dateplot}
\usepackage{xargs}
\usepackage{rotating}
\usepackage{pdflscape}
\usepackage{afterpage}
\usepackage[paper=A4,pagesize]{typearea}
\graphicspath{{../../figures/}} 
\usepackage{subcaption} 
\usepackage{hyperref}
\usepackage{amsmath,amssymb} 
\usepackage{indentfirst}
\usepackage[algo2e,linesnumbered,ruled]{algorithm2e}
\usepackage{algorithmic}
%%If desirable, the user can enable Times New Roman Fonts by uncommenting the next line. G.L.
% \usepackage{mathptmx}
\usepackage{pdfpages}
\usepackage{multirow}
\usepackage{color}
\usepackage{blindtext}
\usepackage{float}
\usepackage{nameref}
\usepackage{cleveref}
\usepackage{multicol}
\usepackage{listings}
\usepackage{enumerate}
\usepackage[acronym,toc]{glossaries}\makeglossaries
\usepackage{tikz}
\usepackage{ladder} %https://github.com/AurelienC/tex-ladder/blob/master/ladder.sty
\usetikzlibrary{arrows,shapes,automata,petri,external,arrows.meta}
	\tikzset{
	place/.style={
	circle,
	thick,
	draw=black!100, % draw=blue!75,
    % fill=blue!20,
    minimum size=6mm
  },
  extPlace/.style={
    circle,
    dotted,
    draw=black!100, % draw=blue!75,
    % fill=blue!20,
    minimum size=6mm
  },
  extTransition/.style={
    rectangle,
    dotted,
    fill=white,
    minimum width=8mm,
    inner ysep=0.7pt
  },
  transition/.style={
    rectangle,
    thick,
    fill=black,
    minimum width=8mm,
    inner ysep=0.7pt
  },
  extTimedtransition/.style={
    rectangle,
    dotted,
    fill=white,
    minimum width=8mm,
    inner ysep=2pt
  },
  timedtransition/.style={
    rectangle,
    thick,
    fill=white,
    minimum width=8mm,
    inner ysep=2pt
  },
  inhibitor/.style={-o},
  text/.style={}
}

\makeatletter
\tikz@def@grow@tokens{2}{1}{-1.5}{0}
\tikz@def@grow@tokens{2}{2}{1.5}{0}
% \tikz@def@grow@tokens{3}{1}{-1}{0}
% \tikz@def@grow@tokens{3}{2}{0}{1}
% \tikz@def@grow@tokens{3}{3}{1.5}{-1}
\makeatother


\definecolor{darkblue}{rgb}{0,0,0.3}
\definecolor{blue}{rgb}{0,0,0.5}
\definecolor{color1}{rgb}{1,0.2,0.3}
\definecolor{color2}{rgb}{0.05490196078,0.41176470588,0.13333333333}
% rgb(14, 105, 34)

\definecolor{color3}{rgb}{0.2,0.2,0.8}
% hyperref setup
\hypersetup{
  % pdftitle={\title},
  pdfauthor={Rafael Accácio Nogueira},
  pdfcreator={Rafael Accácio Nogueira},     
  bookmarksopen=true,         
  bookmarksopenlevel=1,       
  colorlinks=true, % false => boxes 
  linkcolor=blue,
  filecolor=red,  
  urlcolor=blue,  
  citecolor=blue,              
  pdfstartview=Fit,          
  pdfpagemode=UseOutlines,    % this is the option you were lookin for
  pdfpagelayout=TwoPageRight,
}

\makeatletter
\let\stdchapter\chapter
\renewcommand*\chapter{%
  \@ifstar{\starchapter}{\@dblarg\nostarchapter}}
\newcommand*\starchapter[1]{\stdchapter*{#1}}
\def\nostarchapter[#1]#2{
  \stdchapter[#1]{\protect\hyperlink{tocsection}{#1}}}
\makeatother

\makeatletter
\let\stdsection\section
\renewcommand*\section{%
  \@ifstar{\starsection}{\@dblarg\nostarsection}}
\newcommand*\starsection[1]{\stdsection*{#1}}
\def\nostarsection[#1]#2{
  \stdsection[#1]{\protect\hyperlink{tocsection}{#1}}}
\makeatother

\makeatletter
\let\stdsubsection\subsection
\renewcommand*\subsection{%
  \@ifstar{\starsubsection}{\@dblarg\nostarsubsection}}
\newcommand*\starsubsection[1]{\stdsubsection*{#1}}
\def\nostarsubsection[#1]#2{
  \stdsubsection[#1]{\protect\hyperlink{tocsection}{#1}}}
\makeatother

\makeatletter
\let\stdsubsubsection\subsubsection
\renewcommand*\subsubsection{%
  \@ifstar{\starsubsubsection}{\@dblarg\nostarsubsubsection}}
\newcommand*\starsubsubsection[1]{\stdsubsubsection*{#1}}
\def\nostarsubsubsection[#1]#2{
  \stdsubsubsection[#1]{\protect\hyperlink{tocsection}{#1}}}
\makeatother

\let\oldtoc\tableofcontents
\renewcommand{\tableofcontents}{\pagebreak\hypertarget{tocsection}{}\label{tocsection}\oldtoc}


\newcommand{\figplaceholder}[2]{
	\begin{figure}[H]
		\begin{center}	
			\rule{8cm}{8cm}
			\caption{\todo[FORGOT TO INCLUDE FIGURE]{#1 (placeholder)}}
			\label{fig:#2}
		\end{center}
	\end{figure}
}

\newif\ifdebug
\newcommand{\draft}{\debugtrue}
\newcommand{\final}{\debugfalse}
\newcommand{\todo}[2][FORGOT TO DO SOMETHING]{\ifdebug {\color{red}#2}\else \PackageError{}{#1}{}\fi}
\newcommand\doing[1]{\ifdebug {\color{blue}#1}\else \PackageError{}{FORGOT TO DO SOMETHING}{}\fi}
\newcommand\warning[1]{\ifdebug {\color{red}#1}\fi}
\newcommand\note[1]{\ifdebug {\color{orange}#1}\fi}

\usepackage{fancyhdr}
\pagestyle{fancy}

\fancyhead[L]{\warning{DRAFT}}
\fancyhead[R]{\warning{DEBUG ON}}

\fancyfoot[L]{\warning{TURN DEBUG OFF}}
\fancyfoot[R]{\warning{DRAFT}}

\newtheorem{theorem}{Theorem}
\numberwithin{theorem}{chapter}

\newtheorem{example}{Example}
\numberwithin{example}{chapter}

\newtheorem{definition}{Definition}
\numberwithin{definition}{chapter}

\newtheorem{observation}{Remark}
\numberwithin{observation}{chapter}

\usepackage[export]{adjustbox}

\newcommand{\includetikzfigure}[2][]{
    \ifdebug {\includegraphics[#1]{#2.pdf}}
    \else  \includegraphics[#1]{#2}\fi
}

\newcommand{\addtikzfigureLandscape}[4][width=0.8\textwidth]{
\KOMAoptions{paper=landscape}
\recalctypearea
  \vspace*{\fill}
  \begin{figure}[H]
    \centering
    \ifdebug {\includegraphics[#1]{#2.pdf}}
    \else  \includegraphics[#1]{#2}\fi
    \caption{#3}
    \label{fig:#4}
  \end{figure}
  \vspace*{\fill}
\KOMAoptions{paper=portrait}
\recalctypearea
}

\newcommand{\addtikzfigureLandscapeAthree}[4][width=0.8\textwidth]{
\KOMAoptions{paper=a3,paper=landscape}
% \KOMAoptions{paper=landscape}
\recalctypearea
  \begin{figure}[H]
    \vspace{-2cm}
    \centering
    \ifdebug {\centerline{\includegraphics[#1]{#2.pdf}}}
    \else  \centerline{\includegraphics[#1]{#2}}
\fi
    % \caption{#3}
    % \label{fig:#4}
  \end{figure}
\KOMAoptions{paper=a4,paper=portrait}
\recalctypearea
}

% \newcommand{\addtikzfigureVertCent}[3]{
% \KOMAoptions{paper=landscape}
% \recalctypearea
% % \begin{landscape}
% \vspace*{\fill}
%   \begin{figure}[H]
%     % \centering
%     % \resizebox{\hsize}{!}{
%     % \input{#1}
%      \includegraphics[width=1.15\textwidth]{#1}
%     % }
%     \caption{#2}
%     \label{fig:#3}
%   \end{figure}
% \vspace*{\fill}
% % \end{landscape}
% \KOMAoptions{paper=portrait}
% \recalctypearea
% }

\newcolumntype{P}[1]{>{\centering\arraybackslash}p{#1}}
\newcolumntype{M}[1]{>{\centering\arraybackslash}m{#1}}
\definecolor{keywordstyle}{rgb}{0,0,0.82}
\definecolor{commentstyle}{rgb}{0,0.6,0}
\definecolor{numberstyle}{rgb}{0.5,0.5,0.5}
\definecolor{stringstyle}{rgb}{0.58,0,0.82}

% Listing options
\lstset{ 
  % backgroundcolor=\color{white},   % choose the background color; you must add \usepackage{color} or \usepackage{xcolor}; should come as last argument
  basicstyle=\footnotesize,        % the size of the fonts that are used for the code
  breakatwhitespace=false,         % sets if automatic breaks should only happen at whitespace
  breaklines=true,                 % sets automatic line breaking
  captionpos=t,                    % sets the caption-position to bottom
  commentstyle=\color{commentstyle},    % comment style
  deletekeywords={...},            % if you want to delete keywords from the given language
  escapeinside={\%*}{*)},          % if you want to add LaTeX within your code
  extendedchars=true,              % lets you use non-ASCII characters; for 8-bits encodings only, does not work with UTF-8
  % firstnumber=1000,                % start line enumeration with line 1000
  % frame=single,	                   % adds a frame around the code
  keepspaces=true,                 % keeps spaces in text, useful for keeping indentation of code (possibly needs columns=flexible)
  keywordstyle=\color{keywordstyle},       % keyword style
  % language=Octave,                 % the language of the code
  morekeywords={*,...},            % if you want to add more keywords to the set
  numbers=left,                    % where to put the line-numbers; possible values are (none, left, right)
  numbersep=10pt,                   % how far the line-numbers are from the code
  numberstyle=\tiny\color{numberstyle}, % the style that is used for the line-numbers
  rulecolor=\color{black},         % if not set, the frame-color may be changed on line-breaks within not-black text (e.g. comments (green here))
  showspaces=false,                % show spaces everywhere adding particular underscores; it overrides 'showstringspaces'
  showstringspaces=false,          % underline spaces within strings only
  showtabs=false,                  % show tabs within strings adding particular underscores
  stepnumber=2,                    % the step between two line-numbers. If it's 1, each line will be numbered
  stringstyle=\color{stringstyle},     % string literal style
  tabsize=2,	                   % sets default tabsize to 2 spaces
  title=\lstname                   % show the filename of files included with \lstinputlisting; also try caption instead of title
}

%% as seen in https://tex.stackexchange.com/a/183682/143332
\makeatletter
\newcommand\Autoref[1]{\@first@ref#1,@}
\def\@throw@dot#1.#2@{#1}% discard everything after the dot
\def\@set@refname#1{%    % set \@refname to autoefname+s using \getrefbykeydefault
    \edef\@tmp{\getrefbykeydefault{#1}{anchor}{}}%
    \xdef\@tmp{\expandafter\@throw@dot\@tmp.@}%
    \ltx@IfUndefined{\@tmp autorefnameplural}%
         {\def\@refname{\@nameuse{\@tmp autorefname}s}}%
         {\def\@refname{\@nameuse{\@tmp autorefnameplural}}}%
}
\def\@first@ref#1,#2{%
  \ifx#2@\autoref{#1}\let\@nextref\@gobble% only one ref, revert to normal \autoref
  \else%
    \@set@refname{#1}%  set \@refname to autoref name
    \@refname~\ref{#1}% add autoefname and first reference
    \let\@nextref\@next@ref% push processing to \@next@ref
  \fi%
  \@nextref#2%
}
\def\@next@ref#1,#2{%
   \ifx#2@ and~\ref{#1}\let\@nextref\@gobble% at end: print and+\ref and stop
   \else, \ref{#1}% print  ,+\ref and continue
   \fi%
   \@nextref#2%
 }
 \makeatother

\newcommand{\colvec}[2][1]{%
  \scalebox{#1}{%
    \renewcommand{\arraystretch}{.7}%
    $\begin{bmatrix}#2\end{bmatrix}$%
  }
}


%%% Local Variables:
%%% mode: latex
%%% TeX-master: "./monografia.tex"
%%% End:
