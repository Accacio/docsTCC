\usepackage[utf8]{inputenc}
\usepackage[english]{babel}
% \usepackage[brazil]{babel}
\usepackage[T1]{fontenc}
\usepackage{graphicx}
\usepackage{pax}
\usepackage{tikzscale}

\usepackage{rotating}
\usepackage{pdflscape}
\usepackage{afterpage}
\usepackage[paper=A4,pagesize]{typearea}
\graphicspath{{../../figures/}} 
\usepackage{subcaption} 
\usepackage{hyperref}
\usepackage{amsmath,amssymb} 
\usepackage{indentfirst}
\usepackage[algo2e]{algorithm2e}
%%If desirable, the user can enable Times New Roman Fonts by uncommenting the next line. G.L.
% \usepackage{mathptmx}
\usepackage{pdfpages}
\usepackage{multirow}
\usepackage{color}
\usepackage{blindtext}
\usepackage{float}
\usepackage{nameref}
\usepackage{multicol}
\usepackage{listings}
\usepackage{enumerate}
\usepackage[acronym,toc]{glossaries}\makeglossaries
\usepackage{tikz}
\usepackage{ladder} %https://github.com/AurelienC/tex-ladder/blob/master/ladder.sty
\usetikzlibrary{arrows,shapes,automata,petri}

\tikzset{
  place/.style={
    circle,
    thick,
    draw=black!100, % draw=blue!75,
    % fill=blue!20,
    minimum size=6mm
  },
  transition/.style={
    rectangle,
    thick,
    fill=black,
    minimum width=8mm,
    inner ysep=0.7pt
  },
  timedtransition/.style={
    rectangle,
    thick,
    fill=black,
    minimum width=8mm,
    inner ysep=2pt
  },
  inhibitor/.style={-o},
  text/.style={}
}

\definecolor{darkblue}{rgb}{0,0,0.3}
\definecolor{blue}{rgb}{0,0,0.5}


% hyperref setup

\hypersetup{
  % pdftitle={\title},
  pdfauthor={Rafael Accácio Nogueira},
  pdfcreator={Rafael Accácio Nogueira},     
  bookmarksopen=true,         
  bookmarksopenlevel=1,       
  colorlinks=true, % false => boxes 
  linkcolor=blue,
  filecolor=red,  
  urlcolor=blue,  
  citecolor=blue,              
  pdfstartview=Fit,          
  pdfpagemode=UseOutlines,    % this is the option you were lookin for
  pdfpagelayout=TwoPageRight,
}
% \hypersetup{%
%   pdfauthor={\@authname\ \@authsurn},
%   pdftitle={\local@title},
%   pdfsubject={\local@doctype\ de \@degreename\ em \local@deptname\ da COPPE/UFRJ},
%   pdfcreator={LaTeX with CoppeTeX toolkit},
%   breaklinks={true},
%   raiselinks={true},
%   pageanchor={true},
% }}

\makeatletter
\let\stdchapter\chapter
\renewcommand*\chapter{%
  \@ifstar{\starchapter}{\@dblarg\nostarchapter}}
\newcommand*\starchapter[1]{\stdchapter*{#1}}
\def\nostarchapter[#1]#2{
  \stdchapter[#1]{\protect\hyperlink{tocsection}{#1}}}
\makeatother

\makeatletter
\let\stdsection\section
\renewcommand*\section{%
  \@ifstar{\starsection}{\@dblarg\nostarsection}}
\newcommand*\starsection[1]{\stdsection*{#1}}
\def\nostarsection[#1]#2{
  \stdsection[#1]{\protect\hyperlink{tocsection}{#1}}}
\makeatother

\makeatletter
\let\stdsubsection\subsection
\renewcommand*\subsection{%
  \@ifstar{\starsubsection}{\@dblarg\nostarsubsection}}
\newcommand*\starsubsection[1]{\stdsubsection*{#1}}
\def\nostarsubsection[#1]#2{
  \stdsubsection[#1]{\protect\hyperlink{tocsection}{#1}}}
\makeatother

\makeatletter
\let\stdsubsubsection\subsubsection
\renewcommand*\subsubsection{%
  \@ifstar{\starsubsubsection}{\@dblarg\nostarsubsubsection}}
\newcommand*\starsubsubsection[1]{\stdsubsubsection*{#1}}
\def\nostarsubsubsection[#1]#2{
  \stdsubsubsection[#1]{\protect\hyperlink{tocsection}{#1}}}
\makeatother

\let\oldtoc\tableofcontents
\renewcommand{\tableofcontents}{\oldtoc\hypertarget{tocsection}{}\label{tocsection}}

\newcommand{\acr}[3]{\newacronym{#1}{#2}{#3}
  \expandafter\newcommand\csname #1\endcsname{\gls{#1}}}

\newcommand{\symbl}[3]{\newglossaryentry{#1}{name = #2,	description = #3}
  \expandafter\newcommand\csname #1\endcsname{\gls{#1}}
}

\newcommand{\figplaceholder}[2]{
	\begin{figure}[H]
		\begin{center}	
			\rule{8cm}{8cm}
			\caption{\todo[FORGOT TO INCLUDE FIGURE]{#1 (placeholder)}}
			\label{fig:#2}
		\end{center}
	\end{figure}
}


\newif\ifdebug
\newcommand{\draft}{\debugtrue}
\newcommand{\final}{\debugfalse}
\newcommand{\todo}[2][FORGOT TO DO SOMETHING]{\ifdebug {\color{red}#2}\else \PackageError{}{#1}{}\fi}
\newcommand\doing[1]{\ifdebug {\color{blue}#1}\else \PackageError{}{FORGOT TO DO SOMETHING}{}\fi}
\newcommand\warning[1]{\ifdebug {\color{red}#1}\fi}
\newcommand\note[1]{\ifdebug {\color{orange}#1}\fi}

\usepackage{fancyhdr}
\pagestyle{fancy}

\fancyhead[L]{\warning{DRAFT}}
\fancyhead[R]{\warning{DEBUG ON}}

\fancyfoot[L]{\warning{TURN DEBUG OFF}}
\fancyfoot[R]{\warning{DRAFT}}

\newtheorem{example}{Exemplo}
\numberwithin{example}{chapter}

\newtheorem{theorem}{Definição}
\numberwithin{theorem}{chapter}

\newtheorem{observation}{Observação}
%\numberwithin{observation}{chapter}
\usepackage[export]{adjustbox}

\newcommand{\addtikzfigure}[3]{
\KOMAoptions{paper=landscape}
\recalctypearea
  \vspace*{\fill}
  \begin{figure}[H]
    \centering
      \includegraphics[width=0.8\textwidth]{#1}
    \caption{#2}
    \label{fig:#3}
  \end{figure}
  \vspace*{\fill}
\KOMAoptions{paper=portrait}
\recalctypearea
}

% \newcommand{\addtikzfigureVertCent}[3]{
% \KOMAoptions{paper=landscape}
% \recalctypearea
% % \begin{landscape}
% \vspace*{\fill}
%   \begin{figure}[H]
%     % \centering
%     % \resizebox{\hsize}{!}{
%     % \input{#1}
%      \includegraphics[width=1.15\textwidth]{#1}
%     % }
%     \caption{#2}
%     \label{fig:#3}
%   \end{figure}
% \vspace*{\fill}
% % \end{landscape}
% \KOMAoptions{paper=portrait}
% \recalctypearea
% }

%%% Local Variables:
%%% mode: latex
%%% TeX-master: "./monografia.tex"
%%% End:
