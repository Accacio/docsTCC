\usepackage{graphicx}
\usepackage[utf8]{inputenc}
\usepackage[T1]{fontenc}
\usepackage{pgfpages}
\usepackage{multicol}
\usepackage{picture}
% \usepackage{bbding}
% \usepackage[portuguese]{babel}
\uselanguage{Portuguese}
\languagepath{Portuguese}
\setbeamertemplate{caption}[numbered]
% \usepackage{sansmathaccent}
\pdfmapfile{+Sensationsmache.map}
\graphicspath{{../../figures/}}
% \usepackage{movie15}
% \usepackage{media9}
\usepackage{stmaryrd} 
\usepackage{listings}
\definecolor{keywordstyle}{rgb}{0,0,0.82}
\definecolor{commentstyle}{rgb}{0,0.6,0}
\definecolor{numberstyle}{rgb}{0.5,0.5,0.5}
\definecolor{stringstyle}{rgb}{0.58,0,0.82}

% Listing options
\lstset{ 
  % backgroundcolor=\color{white},   % choose the background color; you must add \usepackage{color} or \usepackage{xcolor}; should come as last argument
  basicstyle=\footnotesize,        % the size of the fonts that are used for the code
  breakatwhitespace=false,         % sets if automatic breaks should only happen at whitespace
  breaklines=true,                 % sets automatic line breaking
  captionpos=t,                    % sets the caption-position to bottom
  commentstyle=\color{commentstyle},    % comment style
  deletekeywords={...},            % if you want to delete keywords from the given language
  escapeinside={\%*}{*)},          % if you want to add LaTeX within your code
  extendedchars=true,              % lets you use non-ASCII characters; for 8-bits encodings only, does not work with UTF-8
  % firstnumber=1000,                % start line enumeration with line 1000
  % frame=single,	                   % adds a frame around the code
  keepspaces=true,                 % keeps spaces in text, useful for keeping indentation of code (possibly needs columns=flexible)
  keywordstyle=\color{keywordstyle},       % keyword style
  % language=Octave,                 % the language of the code
  morekeywords={*,...},            % if you want to add more keywords to the set
  numbers=left,                    % where to put the line-numbers; possible values are (none, left, right)
  numbersep=10pt,                   % how far the line-numbers are from the code
  numberstyle=\tiny\color{numberstyle}, % the style that is used for the line-numbers
  rulecolor=\color{black},         % if not set, the frame-color may be changed on line-breaks within not-black text (e.g. comments (green here))
  showspaces=false,                % show spaces everywhere adding particular underscores; it overrides 'showstringspaces'
  showstringspaces=false,          % underline spaces within strings only
  showtabs=false,                  % show tabs within strings adding particular underscores
  stepnumber=2,                    % the step between two line-numbers. If it's 1, each line will be numbered
  stringstyle=\color{stringstyle},     % string literal style
  tabsize=2,	                   % sets default tabsize to 2 spaces
  title=\lstname                   % show the filename of files included with \lstinputlisting; also try caption instead of title
}

\usepackage{tikzscale}
\usepackage{tikz}
\usepackage{adjustbox}
\usepackage{pgfplots}
% \newcommand{\bulletpoint}[1]{\begin{itemize}
% 		\item #1
% \end{itemize}}

\newcommand{\bulletpoint}[1]{$\bullet$ #1}
\newif\ifdebug
\newcommand{\draft}{\debugtrue}
\newcommand{\final}{\debugfalse}
\newcommand{\includetikzfigure}[2][]{
    \ifdebug {\includegraphics[#1]{#2.pdf}}
    \else  \includegraphics[#1]{#2}\fi
}

\pgfplotsset{compat=newest}
\usepgfplotslibrary{groupplots}
\usepgfplotslibrary{dateplot}
% \newtheorem{theorem}{Teorema}
% \numberwithin{theorem}{section}

% \newtheorem{example}{Exemplo}
% \numberwithin{example}{section}

% \newtheorem{definition}{Definição}
% \numberwithin{definition}{section}

% \newtheorem{observation}{Observação}
% \numberwithin{observation}{section}

\newlength{\ladderskip}
\newlength{\ladderrungsep}
\usetikzlibrary{patterns}
\usetikzlibrary{arrows,shapes,circuits.plc.ladder,external}
\usetikzlibrary{arrows,shapes,automata,petri,external,arrows.meta}
\tikzset{
	place/.style={
	circle,
	thick,
	% draw=black!100,
  % draw=blue!75,
    % fill=blue!20,
    minimum size=6mm
  },
  extPlace/.style={
    circle,
    dotted,
    % draw=black!100, % draw=blue!75,
    % fill=blue!20,
    minimum size=6mm
  },
  extTransition/.style={
    rectangle,
    dotted,
    fill=white,
    minimum width=8mm,
    inner ysep=0.7pt
  },
  transition/.style={
    rectangle,
    thick,
    fill=black,
    minimum width=8mm,
    inner ysep=0.7pt
  },
  extTimedtransition/.style={
    rectangle,
    dotted,
    fill=white,
    minimum width=8mm,
    inner ysep=2pt
  },
  timedtransition/.style={
    rectangle,
    thick,
    fill=white,
    minimum width=8mm,
    inner ysep=2pt
  },
  inhibitor/.style={-o},
  text/.style={}
}
\makeatletter
\tikz@def@grow@tokens{2}{1}{-1.}{0}
\tikz@def@grow@tokens{2}{2}{1.}{0}
% \tikz@def@grow@tokens{3}{1}{-1}{0}
% \tikz@def@grow@tokens{3}{2}{0}{1}
% \tikz@def@grow@tokens{3}{3}{1.5}{-1}
\makeatother
\newcommand{\colvec}[2][1]{%
  \scalebox{#1}{%
    \renewcommand{\arraystretch}{.7}%
    $\begin{bmatrix}#2\end{bmatrix}$%
  }
}












%%% Local Variables:
%%% mode: latex
%%% TeX-master: "presentation"
%%% End:
