\usepackage{graphicx}
\usepackage[utf8]{inputenc}
\usepackage[T1]{fontenc}
\usepackage{pgfpages}
\usepackage{multicol}
\usepackage{picture}
\usepackage{bbding}
% \usepackage[portuguese]{babel}
\uselanguage{Portuguese}
\languagepath{Portuguese}
\setbeamertemplate{caption}[numbered]
\usepackage{sansmathaccent}
\pdfmapfile{+Sensationsmache.map}
\graphicspath{{../../figures/}}
% \usepackage{movie15}
% \usepackage{media9}
\usepackage{stmaryrd} 

\usepackage{tikzscale}
\usepackage{tikz}
\usepackage{adjustbox}
\usepackage{pgfplots}
% \newcommand{\bulletpoint}[1]{\begin{itemize}
% 		\item #1
% \end{itemize}}

\newcommand{\bulletpoint}[1]{$\bullet$ #1}
\newif\ifdebug
\newcommand{\draft}{\debugtrue}
\newcommand{\final}{\debugfalse}
\newcommand{\includetikzfigure}[2][]{
    \ifdebug {\includegraphics[#1]{#2.pdf}}
    \else  \includegraphics[#1]{#2}\fi
}

\pgfplotsset{compat=newest}
\usepgfplotslibrary{groupplots}
\usepgfplotslibrary{dateplot}
% \newtheorem{theorem}{Teorema}
% \numberwithin{theorem}{section}

% \newtheorem{example}{Examplo}
% \numberwithin{example}{section}

% \newtheorem{definition}{Definição}
% \numberwithin{definition}{section}

% \newtheorem{observation}{Observação}
% \numberwithin{observation}{section}

\newlength{\ladderskip}
\newlength{\ladderrungsep}
\usetikzlibrary{patterns}
\usetikzlibrary{arrows,shapes,circuits.plc.ladder,external}
\usetikzlibrary{arrows,shapes,automata,petri,external,arrows.meta}
\tikzset{
	place/.style={
	circle,
	thick,
	% draw=black!100,
  % draw=blue!75,
    % fill=blue!20,
    minimum size=6mm
  },
  extPlace/.style={
    circle,
    dotted,
    % draw=black!100, % draw=blue!75,
    % fill=blue!20,
    minimum size=6mm
  },
  extTransition/.style={
    rectangle,
    dotted,
    fill=white,
    minimum width=8mm,
    inner ysep=0.7pt
  },
  transition/.style={
    rectangle,
    thick,
    fill=black,
    minimum width=8mm,
    inner ysep=0.7pt
  },
  extTimedtransition/.style={
    rectangle,
    dotted,
    fill=white,
    minimum width=8mm,
    inner ysep=2pt
  },
  timedtransition/.style={
    rectangle,
    thick,
    fill=white,
    minimum width=8mm,
    inner ysep=2pt
  },
  inhibitor/.style={-o},
  text/.style={}
}
\makeatletter
\tikz@def@grow@tokens{2}{1}{-1.}{0}
\tikz@def@grow@tokens{2}{2}{1.}{0}
% \tikz@def@grow@tokens{3}{1}{-1}{0}
% \tikz@def@grow@tokens{3}{2}{0}{1}
% \tikz@def@grow@tokens{3}{3}{1.5}{-1}
\makeatother
\newcommand{\colvec}[2][1]{%
  \scalebox{#1}{%
    \renewcommand{\arraystretch}{.7}%
    $\begin{bmatrix}#2\end{bmatrix}$%
  }
}












%%% Local Variables:
%%% mode: latex
%%% TeX-master: "presentation"
%%% End:
