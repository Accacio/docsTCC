
\section{Conclusões}

\begin{frame}{Conclusões}
\begin{itemize}\pause
\item Método para controle, observação e identificação\pause 
\item DAOCT funciona mas depende que bons caminhos tenham sido dados\pause 
\end{itemize}
\begin{block}{Trabalhos Futuros}
\begin{itemize}
\item Comparar o comportamento: monolítico $\times$ dividido em módulos. \pause
\item Método para escolher o melhor vetor inicial\pause 
\item Método que não dependa da escolha do vetor inicial
\end{itemize}
\end{block}
\note{
  \begin{itemize}
\item Foi apresentado um Método para ...
\item depende de uma boa escolha de vetor inicial para representar bem o sistema
\end{itemize}
\begin{block}{Trabalhos Futuros}
\begin{itemize}
\item Comparar o comportamento do sistema monolítico, apresentado nesse nesse
  trabalho com a observação dos seus módulos. \pause
\end{itemize}
\end{block}
}
\end{frame}

\begin{frame}{}
  \Large{Muito Obrigado!}\\
  \large{\quad Perguntas?}
\begin{center}
\begin{itemize}
\item \small contato: raccacio@poli.ufrj.br
\end{itemize}
\end{center}
\footnotesize{ Apresentação disponível em
  https://github.com/Accacio/docsTCC/raw/master/presentation.pdf}\\
\includegraphics[height=0.2\textheight]{presentation.png}
\end{frame}
%%% Local Variables:
%%% mode: latex
%%% TeX-master: "../presentation"
%%% End: